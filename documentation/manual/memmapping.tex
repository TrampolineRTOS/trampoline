%!TEX encoding = UTF-8 Unicode
%!TEX root = ./main.tex

\chapter{Memory mapping}

\lettrine{T}he \autosar\ consortium has defined a set of macros\,\cite{autosar44CompilerAbstraction} in order to adapt the memory mapping directives to the different existing compilers. Indeed, memory mapping directives are not part of the C language and it is therefore impossible to write portable code between different compilers without going through this set of macros. In addition, some MCUs have a segmented memory model and require additional pointer directives to specify whether the pointer is to data in the same segment or to data in a different segment. In the first case, it is a \emph{near} pointer (usually stored in a 16 bits word). In the second case it is a \emph{far} pointer (usually stored in a 24 bits word).

It remains that these macros are not particularly intuitive in their use and require some explanations that we will give here.

\section{Memory mapping directives}

Memory mapping consists in assigning to each object of the application (variables, constants, functions) and of the operating system a named memory area where the object will be stored. Memory mapping directives take various forms depending on the compiler. For example, putting a function named \cfunction{f} in the memory area \ctype{.osCode} when using \tool{gcc} is done as follows:

\begin{lstlisting}[language=C]
void __attribute__ ((section (".osCode"))) f() { ... }
\end{lstlisting}

while doing the same thing using Freescale's CodeWarrior compiler (previously Metrowerks) requires the following directive:

\begin{lstlisting}[language=C]
#pragma section code_type ".osCode"
void f() { ... }
\end{lstlisting}

\autosar\ defines several macros to encapsulate these directives and these macros work with the declarations of the memory sections.

\section{The memory sections}

For each task and ISR declared in the OIL file, Goil generates several memory sections. These sections are selected via macro definitions with names of the form \cmacro{APP_Task_<name>_START_SEC_<section_type>} and \cmacro{APP_Task_<name>_STOP_SEC_<section_type>} for tasks and \cmacro{APP_ISR_<name>_START_SEC_<section_type>} and \cmacro{APP_ISR_<name>_STOP_SEC_<section_type>} for ISRs. \cmacro{<name>} is the name of the task or ISR and \cmacro{<section_type>} is the type of section. The section types are as follows:

\begin{description}

\item[\cmacro{CODE}] is the section used for the process code and for the functions called by the process. If, for example, the task t1 is declared in the OIL file, its code will be written in C as follows:
\begin{lstlisting}[language=C]
#define APP_Task_t1_START_SEC_CODE
#include "MemMap.h"
TASK(t1)
{
  ...
  TerminateTask();
}
#define APP_Task_t1_STOP_SEC_CODE
#include "MemMap.h"
\end{lstlisting}

\item[\cmacro{STACK}] is the section used for the process stack. This section is used in the files generated by \goil.

\item[\cmacro{VAR_<init_policy>_<alignment>}] are the sections used for process globals or static variables. \cmacro{<init_policy>} can take the following values:
  \begin{description}
  \item[\cmacro{NOINIT}] for uninitialized variables.
  \item[\cmacro{POWER_ON_INIT}] for variables initialized at MCU startup.
  \end{description}
\cmacro{<alignment>} can take the following values:
  \begin{description}
  \item[\cmacro{32BIT}] for 4 bytes alignment.
  \item[\cmacro{16BIT}] for 2 bytes alignment.
  \item[\cmacro{8BIT}] for 1 byte alignment.
  \item[\cmacro{UNSPECIFIED}] for data sizes that do not fit into any of the other categories.
  \end{description}

\item[\cmacro{CONST_<alignment>}] are the sections used for process globals constants.
\end{description}

For example, if task t1 uses two 8-bit constants, \var{c1} and \var{c2}, and one 32-bit variable, \var{v1}, uninitialized, they will be declared as follows:

\begin{lstlisting}[language=C]
#define APP_Task_t1_START_SEC_CONST_8BIT
#include "MemMap.h"
CONST(uint8, AUTOMATIC) c1 = 3;
CONST(uint8, AUTOMATIC) c2 = 7;
#define APP_Task_t1_STOP_SEC_CONST_8BIT
#include "MemMap.h"

#define APP_Task_t1_START_SEC_VAR_NOINIT_32BIT
#include "MemMap.h"
VAR(uint8, AUTOMATIC) v1;
#define APP_Task_t1_STOP_SEC_VAR_NOINIT_32BIT
#include "MemMap.h"
\end{lstlisting}



\begin{table}[htp]
\caption{Sections generated for task t1}
\begin{center}\scriptsize
\begin{tabular}{|c|c|}
\hline
APP_Task_toto_START_SEC_CODE & APP_Task_toto_STOP_SEC_CODE \\
APP_Task_toto_START_SEC_STACK & APP_Task_toto_STOP_SEC_STACK \\
APP_Task_toto_START_SEC_VAR_NOINIT_32BIT & APP_Task_toto_STOP_SEC_VAR_NOINIT_32BIT \\
APP_Task_toto_START_SEC_VAR_NOINIT_16BIT & APP_Task_toto_STOP_SEC_VAR_NOINIT_16BIT \\
APP_Task_toto_START_SEC_VAR_NOINIT_8BIT & APP_Task_toto_STOP_SEC_VAR_NOINIT_8BIT \\
APP_Task_toto_START_SEC_VAR_NOINIT_BOOLEAN & APP_Task_toto_STOP_SEC_VAR_NOINIT_BOOLEAN \\
APP_Task_toto_START_SEC_VAR_NOINIT_UNSPECIFIED & APP_Task_toto_STOP_SEC_VAR_NOINIT_UNSPECIFIED \\
APP_Task_toto_START_SEC_VAR_POWER_ON_INIT_32BIT & APP_Task_toto_STOP_SEC_VAR_POWER_ON_INIT_32BIT \\
APP_Task_toto_START_SEC_VAR_POWER_ON_INIT_16BIT & APP_Task_toto_STOP_SEC_VAR_POWER_ON_INIT_16BIT \\
APP_Task_toto_START_SEC_VAR_POWER_ON_INIT_8BIT & APP_Task_toto_STOP_SEC_VAR_POWER_ON_INIT_8BIT \\
APP_Task_toto_START_SEC_VAR_POWER_ON_INIT_BOOLEAN & APP_Task_toto_STOP_SEC_VAR_POWER_ON_INIT_BOOLEAN \\
APP_Task_toto_START_SEC_VAR_POWER_ON_INIT_UNSPECIFIED & APP_Task_toto_STOP_SEC_VAR_POWER_ON_INIT_UNSPECIFIED \\
APP_Task_toto_START_SEC_VAR_FAST_32BIT & APP_Task_toto_STOP_SEC_VAR_FAST_32BIT \\
APP_Task_toto_START_SEC_VAR_FAST_16BIT & APP_Task_toto_STOP_SEC_VAR_FAST_16BIT \\
APP_Task_toto_START_SEC_VAR_FAST_8BIT & APP_Task_toto_STOP_SEC_VAR_FAST_8BIT \\
APP_Task_toto_START_SEC_VAR_FAST_BOOLEAN & APP_Task_toto_STOP_SEC_VAR_FAST_BOOLEAN \\
APP_Task_toto_START_SEC_VAR_FAST_UNSPECIFIED & APP_Task_toto_STOP_SEC_VAR_FAST_UNSPECIFIED \\
APP_Task_toto_START_SEC_VAR_32BIT & APP_Task_toto_STOP_SEC_VAR_32BIT \\
APP_Task_toto_START_SEC_VAR_16BIT & APP_Task_toto_STOP_SEC_VAR_16BIT \\
APP_Task_toto_START_SEC_VAR_8BIT & APP_Task_toto_STOP_SEC_VAR_8BIT \\
APP_Task_toto_START_SEC_VAR_BOOLEAN & APP_Task_toto_STOP_SEC_VAR_BOOLEAN \\
APP_Task_toto_START_SEC_VAR_UNSPECIFIED & APP_Task_toto_STOP_SEC_VAR_UNSPECIFIED \\
APP_Task_toto_START_SEC_CONST_32BIT & APP_Task_toto_STOP_SEC_CONST_32BIT \\
APP_Task_toto_START_SEC_CONST_16BIT & APP_Task_toto_STOP_SEC_CONST_16BIT \\
APP_Task_toto_START_SEC_CONST_8BIT & APP_Task_toto_STOP_SEC_CONST_8BIT \\
APP_Task_toto_START_SEC_CONST_BOOLEAN & APP_Task_toto_STOP_SEC_CONST_BOOLEAN \\
APP_Task_toto_START_SEC_CONST_UNSPECIFIED & APP_Task_toto_STOP_SEC_CONST_UNSPECIFIED \\
APP_Task_toto_START_SEC_CALIB_32BIT & APP_Task_toto_STOP_SEC_CALIB_32BIT \\
APP_Task_toto_START_SEC_CALIB_16BIT & APP_Task_toto_STOP_SEC_CALIB_16BIT \\
APP_Task_toto_START_SEC_CALIB_8BIT & APP_Task_toto_STOP_SEC_CALIB_8BIT \\
APP_Task_toto_START_SEC_CALIB_BOOLEAN & APP_Task_toto_STOP_SEC_CALIB_BOOLEAN \\
APP_Task_toto_START_SEC_CALIB_UNSPECIFIED & APP_Task_toto_STOP_SEC_CALIB_UNSPECIFIED \\
APP_Task_toto_START_SEC_CARTO_32BIT & APP_Task_toto_STOP_SEC_CARTO_32BIT \\
APP_Task_toto_START_SEC_CARTO_16BIT & APP_Task_toto_STOP_SEC_CARTO_16BIT \\
APP_Task_toto_START_SEC_CARTO_8BIT & APP_Task_toto_STOP_SEC_CARTO_8BIT \\
APP_Task_toto_START_SEC_CARTO_BOOLEAN & APP_Task_toto_STOP_SEC_CARTO_BOOLEAN \\
APP_Task_toto_START_SEC_CARTO_UNSPECIFIED & APP_Task_toto_STOP_SEC_CARTO_UNSPECIFIED \\
APP_Task_toto_START_SEC_CONFIG_DATA_32BIT & APP_Task_toto_STOP_SEC_CONFIG_DATA_32BIT \\
APP_Task_toto_START_SEC_CONFIG_DATA_16BIT & APP_Task_toto_STOP_SEC_CONFIG_DATA_16BIT \\
APP_Task_toto_START_SEC_CONFIG_DATA_8BIT & APP_Task_toto_STOP_SEC_CONFIG_DATA_8BIT \\
APP_Task_toto_START_SEC_CONFIG_DATA_BOOLEAN & APP_Task_toto_STOP_SEC_CONFIG_DATA_BOOLEAN \\
\end{tabular}
\end{center}
\label{tab:memsections}
\end{table}%
