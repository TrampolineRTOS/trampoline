\chapter{Tracing the execution}

\section{Introduction}

\section{Events}

The list of events that can be recorded is given below.
All events are timestamped.
Exact format and accuracy of the timestamp depends on the target specific
backend (see section~\ref{sec:trace:targetbackend} below).

\begin{description}
  \item[PROC\_STATE\_CHANGE]: state of a process is changed.
    \begin{itemize}
      \item \texttt{proc_id}: identifier of the process.
      \item \texttt{target_state}: new state of the proc. %TODO: list states
    \end{itemize}

  \item[RES\_STATE\_CHANGE]: state of a resource is changed.
    \begin{itemize}
      \item \texttt{res_id}: identifier of the resource.
      \item \texttt{target_state}: new state of the resource.
    \end{itemize}

  \item[EVENT_SET]: a process sets an event of another process.
    \begin{itemize}
      \item \texttt{proc_id}: owner of the events.
      \item \texttt{ev_id}: list of event set.
    \end{itemize}

  \item[EVENT_RESET]: a process resets one of its events of another process.
    \begin{itemize}
      \item \texttt{ev_id}: list of event reset.
    \end{itemize}

    % TODO: because of AUTOSTART, not straightforward to add proc_id here
  \item[TIMEOBJ_STATE_CHANGE]: state of a timeobj is changed.
    \begin{itemize}
      \item \texttt{timeobj_id}: identifier of the timeobj.
      \item \texttt{target_state}: new state of the timeobj.
    \end{itemize}

  \item[TIMEOBJ_EXPIRE]: an alarm reaches its expiry point.
    \begin{itemize}
      \item \texttt{alarm_id}: identifier of the alarm.
    \end{itemize}

\end{description}

\section{Implementation}

\section{Implementing target specific backends}
\label{sec:trace:targetbackend}
