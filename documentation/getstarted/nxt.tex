
\subsubsection{Lego Mindstorm NXT2.0}

%%%%%%%%%%%%%%%%%%%%%%%%%
\paragraph{Nexttool + Lego Drivers}
%%%%%%
\subparagraph{MAC OS}


Download and install the Lego Drivers (\href{http://mindstorms.lego.com/en-us/support/files/default.aspx#Driver}{http://mindstorms.lego.com/en-us/support/files/default.aspx
\#Driver}) and the firmware update (\href{http://mindstorms.lego.com/en-us/support/files/default.aspx#Firmware}{http://mindstorms.lego.com/en-us/support/files/default.aspx\#Firmware}) for MAC OS. \\
Download Nexttool (\href{http://bricxcc.sourceforge.net/utilities.html}{http://bricxcc.sourceforge.net/utilities.html}) and a new firmware (\href{http://bricxcc.sourceforge.net/lms_arm_jch.zip}{http://bricxcc.sourceforge.net/lms\_arm\_jch.zip}) and update the firmware as explained below :
\begin{itemize}
\item Reset the NXT : To go into firmware update mode, press the reset button (at the back of the NXT, upper left corner beneath the USB connector) for more than 5 seconds while the NXT is turned on. The NXT will audibly tick when it is in firmware update mode.
\item Copy an Enhanced NXT firmware (i.e. lms\_arm\_nbcnxc\_107.rfw) to NeXTTool extracted directory.
\item Launch Nexttool, and updload the Enhanced NXT firmware to the NXT (clicking on "Download firmware"), selecting it.
	\begin{figure}[h] %  figure placement: here, top, bottom, or page
   		\centering
		\includegraphics[width=0.7\textwidth]{pictures/firmware.jpg}
	\end{figure}
\item Remove the battery from the NXT and insert it again, and then press orange rectangle button on the NXT to turn on the Enhanced NXT firmware. The Enhanced NXT firmware has same GUI as the LEGO standard firmware.
\end{itemize}

%%%%
\subparagraph{Linux}
Note: Nextool binary for Linux seems to fail for firmware upload.\\
Install required packages:
\begin{verbatim}
$sudo apt-get install scons libusb-dev libusb-0.1-4
\end{verbatim}
Download the libnxt (\href{http://libnxt.googlecode.com/files/libnxt-0.3.tar.gz}{http://libnxt.googlecode.com/files/libnxt-0.3.tar.gz}) archive and extract it. Go in the new directory and build the project with scons:
\begin{verbatim}
$ cd libnxt-0.3/ 
$ scons
\end{verbatim}

A program call fwflash is created.\\
Download John Hansen's Enhanced NXT firmware (\href{http://bricxcc.sourceforge.net/lms_arm_jch.zip}{http://bricxcc.sourceforge.net/lms\_arm\_jch.zip}) (any version numbered 106 or later includes the native-invocation feature) and store the Enhanced NXT firmware (i.e lms\_arm\_nbcnxc\_1xx.rfw) in the directory where fwflash is stored.\\
Connect the NXT brick to usb and turn it on. Then press the reset button for more than 4s to put it in firmware upload mode (nxt display is cleared but it makes a ticking sound).\\
Flash the firmware with the following command (it takes some dozen of seconds), where 1xx is replaced by the number of the firmware:
\begin{verbatim}
$ sudo ./fwflash lms_arm_nbcnxc_1xx.rfw
\end{verbatim}

Troubleshooting: After completion of the upload, sometimes NXT display is messed and block : reboot it by a quick push on the reset button or remove the battery. If the NXT makes a ticking sound, it is still in firmware upload mode. If troubles, use and see windows firmware update procedure (and LEGO UserGuide).\\

%%%%
\subparagraph{Windows}
Download and install the Lego Drivers (\href{http://mindstorms.lego.com/en-us/support/files/default.aspx#Driver}{http://mindstorms.lego.com/en-us/support/files/default.aspx
\#Driver}) for PC. \\
Download Nexttool (\href{http://bricxcc.sourceforge.net/nexttool.zip}{http://bricxcc.sourceforge.net/nexttool.zip}) and a new firmware (\href{http://bricxcc.sourceforge.net/lms_arm_jch.zip}{http://bricxcc.sourceforge.net/lms\_arm\_jch.zip}) and update the firm-ware as explained below :
\begin{itemize}
\item Reset the NXT : To go into firmware update mode, press the reset button (at the back of the NXT, upper left corner beneath the USB connector) for more than 5 seconds while the NXT is turned on. The NXT will audibly tick when it is in firmware update mode.
\item Copy an Enhanced NXT firmware (i.e. lms\_arm\_nbcnxc\_107.rfw) to NeXTTool extracted directory.
\item Execute Cygwin and type the following command to change the current directory to the NexTTool extracted directory. (NeXTTool is assumed to be extracted under C:$\backslash$cygwin$\backslash$nexttool directory)
	\begin{verbatim}
	$cd C:\cygwin\nexttool
	\end{verbatim}
\item Connect PC and the NXT by USB cable.
\item Type the following command in Cygwin to upload the Enhanced NXT firmware to the NXT (Program upload may take around half minutes and then, NXT LCD is turned to display some chunk from blank).
	\begin{verbatim}
	$./NeXTTool.exe /COM=usb -firmware=lms\_arm\_nbcnxc\_107.rfw
	\end{verbatim}
\item Remove the battery from the NXT and insert it again, and then press orange rectangle button on the NXT to turn on the Enhanced NXT firmware. The Enhanced NXT firmware has same GUI as the LEGO standard firmware.
\end{itemize}




%%%%%%%%%%%%%%%%%%%%%%%%%
\paragraph{Upload a program}
\subparagraph{MAC OS}
To upload a program in the NXT (the nxt example examples/arm/nxt/simple/nxt\_simple\_exe.rxe)
\begin{itemize}
\item Connect the PC and the NXT by USB cable.
\item Launch Nexttool, select "usb port".
	\begin{figure}[htbp] %  figure placement: here, top, bottom, or page
   		\centering
		\includegraphics[width=0.25\textwidth]{pictures/usbport.jpg}
	\end{figure}
\item Go to "NXT Explorer"
	\begin{figure}[htbp] %  figure placement: here, top, bottom, or page
   		\centering
		\includegraphics[width=.7\textwidth]{pictures/nxtexplorer.jpg}
	\end{figure}
\item Click on the "Download selected files to the NXT" and select the nxt\_simple\_exe.rxe file.
\item If program upload was succeeded, you can see the nxt\_simple\_exe.rxe file in the files list as below.
	\begin{center}[h] %  figure placement: here, top, bottom, or page
   		%\centering
		\includegraphics[width=1\textwidth]{pictures/downloadfile.jpg}
	\end{center}
\item To execute a program on the NXT, go in "My files"/"Software files".
\end{itemize}

%%%%
\subparagraph{Linux}
Required Packages: libusb-0.1-4\\
Download this executable of John Hansen's NeXTTool (\href{http://bricxcc.sourceforge.net/nexttool.zip}{http://bricxcc.sourceforge.net/nexttool.zip}) (built from bricxcc svn repository, revision 1)\\
Check the version of NeXTTool, it should be 1.0.1.0:
\begin{verbatim}
$ sudo ./[NEXTTOOL_PATH]/NeXTTool
nexttool version 1.0.1.0 (1.0.1.0)
Copyright (c) 2006, John Hansen
Use "NeXTTool -help" for more information.
\end{verbatim}

To ubload over usb, turn on the NXT, connect it to USB and run the following command (example: nxt\_simple\_exe.rxe) :
\begin{verbatim}
$ sudo ./[NEXTTOOL_PATH]/NeXTTool /COM=usb -download=nxt\_simple\_exe.rxe
\end{verbatim}

Troubleshooting: Test the following command to see if Nexttool is working (set execution right for NeXTTool):
\begin{verbatim}
$ sudo ./[NEXTTOOL_PATH]/NeXTTool /COM=usb -versions
Protocol version = 1.124
Firmware version = 1.xx
\end{verbatim}

To ubload over bluetooth, you need to define an alias name in a file 'nxt.dat' as explained in this post: Minsdtorm 2.0 development on linux . Then turn on the NXT and run the following command (example: nxt\_simple\_exe.rxe) :
\begin{verbatim}
$ sudo ./[NEXTTOOL_PATH]/NeXTTool /COM=alias_bt -download=nxt\_simple\_exe.rxe
\end{verbatim}

To execute a program on the NXT, go in "My files"/"Software files".

%%%%
\subparagraph{Windows}
To upload a program in the NXT (the nxt example examples/arm/nxt/lonely\_exe.rxe) follow the steps below :
\begin{itemize}
\item Connect the PC and the NXT by USB cable.
\item Type the following command in Cygwin (from examples/arm/nxt) :
	\begin{verbatim}
	$./[NEXTTOOL_PATH]/NeXTTool.exe /COM=usb -download=lonely_exe.rxe
	$./[NEXTTOOL_PATH]/NeXTTool.exe /COM=usb -listfiles=lonely_exe.rxe
	\end{verbatim}
\item If program upload was succeeded, program size could be displayed in Cygwin such as the second line in the below command outputs. 
	\begin{verbatim}
	Executing NeXTTool to upload helloworld.rxe...
	helloworld.rxe=15280
	NeXTTool is terminated.
	\end{verbatim}
\item To execute a program on the NXT, go in "My files"/"Software files".
\end{itemize}
