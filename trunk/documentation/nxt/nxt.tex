%\documentclass[10pt]{article}
%\usepackage[utf8]{inputenc}
%\usepackage[english]{}

%\usepackage{color} %For interrogation about report or code
%\usepackage{hyperref} %Liens hypertexte du sommaire
%\usepackage{listings}
%\usepackage{moreverb}
%\usepackage{verbatim}
%\usepackage{graphicx} %To include image

%\title{Trampoline on LEGO MINDSTORMS NXT2 (from zero)}
%\author{Florent PAVIN}

%
%\begin{document}

%\maketitle
%\tableofcontents

%\newpage

\chapter{Trampoline on NXT}

%%%%%%%%%%%%%%%%%
%\section{Introduction}
This document describes how to get started a Trampoline \footnote{Trampoline is an open source RTOS which, once certified, could be compliant with the OSEK/VDX specification. Currently it is not the case, so while Trampoline has the same API as OSEK/VDX, it is not officially compliant. Trampoline is available under the GNU Lesser General Public License V2 and \textbf{is maintained by gcc-4.0.1}. Trampoline runs on several platforms like POSIX, C166 with Keil compiler, Star12X (courtesy of Geensys), PowerPC, ARM (NXT2 for example)} application on Lego Mindstorms NXT2.\\
The following steps will be overviewed :
\begin{itemize}
\item Trampoline installation
\item GCC compilation for ARM platform
\item Installation of drivers and softwares to be able to upload programs on the NXT.
\end{itemize}
This document is explained from a UNIX side. If you are on a Windows system, download and install Cygwin as it is explained on NXTOSEK website (\href{http://lejos-osek.sourceforge.net/installation_windows.htm}{http://lejos-osek.sourceforge.net/installation\_windows.htm}) first.

%%%%%%%%%%%%%%%
\section{Trampoline}
To download Trampoline, type in a terminal (login:anonymous - password:anonymous) :
	\begin{verbatim}
	$svn checkout https://trampoline.rts-software.org/svn/trunk
	\end{verbatim}
Download the libpm (\href{http://galgas.rts-software.org/download/}{http://galgas.rts-software.org/download/}) for your achitecture and copy it in trunk/.\\
To compile GOIL, go in goil/makefile\_[ARCH] depending on your architecture and type in a terminal (you can add -j2 if you have 2 processors) :
	\begin{verbatim}
	$make goil
	\end{verbatim}
This will generate the GOIL executable you'll need. You can add the path to your GOIL (in $\sim$/.profile), adding :
	\begin{verbatim}
	export PATH=[TRAMPOLINE_REPOSITORY]/goil/makefile_[ARCH]/:$PATH
	\end{verbatim}

You can also do it typing "\$sudo make install goil" (this will copy the goil executable in /usr/local/bin).

You can now use Trampoline on Unix system by executing the tests as described in Annex \ref{tests}.

%%%%%%%%%%%%%%%%
\section{GNUARM}
\subsection{MAC OS}
Download and unpack the necessary packages: binutils, gcc, newlib and gdb.
\begin{verbatim}
$ mkdir ~/crossgcc && cd ~/crossgcc
$ wget ftp://sourceware.org/pub/binutils/snapshots/binutils-2.18.50.tar.bz2
$ tar jxf binutils-2.18.50.tar.bz2
$ wget http://ftp.gnu.org/pub/gnu/gcc/gcc-4.2.3/gcc-4.2.3.tar.bz2
$ tar jxf gcc-4.2.3.tar.bz2
$ wget ftp://sources.redhat.com/pub/newlib/newlib-1.16.0.tar.gz
$ tar zxf newlib-1.16.0.tar.gz
$ wget http://ftp.gnu.org/pub/gnu/gdb/gdb-6.6.tar.gz
$ tar zxf gdb-6.6.tar.gz
\end{verbatim}

The installation directory should be /usr/local/crossgcc.

\begin{verbatim}
$ sudo mkdir /usr/local/crossgcc
$ sudo chmod 777 /usr/local/crossgcc
\end{verbatim}

First we build the binutils:

\begin{verbatim}
$ mkdir build-binutils && cd build-binutils
$ ../binutils-2.18.50/configure --target=arm-elf \
--prefix=/usr/local/crossgcc/ 2>&1 | tee configure.log
$ make all install 2>&1 | tee make.log
$ export PATH=$PATH:/usr/local/crossgcc/bin
\end{verbatim}

Build the gcc compiler with C/C++ support:

\begin{verbatim}
$ cd ../gcc-4.2.3
$ ln -s ../newlib-1.16.0/newlib .
$ ln -s ../newlib-1.16.0/libgloss .
$ cd ..
$ mkdir build-gcc && cd build-gcc
$ ../gcc-4.2.3/configure --target=arm-elf \
--prefix=/usr/local/crossgcc/ --with-newlib \
--with-gnu-as --with-gnu-ld --enable-languages=c,c++ 2>&1 | tee configure.log
$ make all install 2>&1 | tee make.log
\end{verbatim}

Build the gdb debugger:

\begin{verbatim}
$ cd ..
$ mkdir build-gdb && cd build-gdb
$ ../gdb-6.6/configure --target=arm-elf --prefix=/usr/local/crossgcc/
$ make all install 2>&1 | tee make.log
\end{verbatim}

Add arm-elf tp your path, adding
\begin{verbatim}
export PATH=/usr/local/crossgcc/bin:$PATH
\end{verbatim}
into $\sim$/.profile

You can now compile an application for ARM as described in Annex \ref{compileanapplication}.

\subsection{Linux}
The complete process should take around 1 hour and finish with the message "Build complete !". \\
Alternatively, you could use a pre-build version of this toolchain (see below).\\

Install the required libraries by running the following command:
\begin{verbatim}
$ sudo apt-get install tk-dev ncurses-dev libmpfr-dev texinfo
\end{verbatim}
on ubuntu 8.10 only: gcc-4.2 (and dependencies) is also required.\\

Download the script build_arm_toolchain.sh (\href{http://lejos-osek.sourceforge.net/installation_linux_files/build_arm_toolchain.sh}{http://lejos-osek.sourceforge.net/installation_linux_files/build_arm_toolchain.sh}). This script is a modified version of the toolchain build script from the NxOS project. It will:
\begin{itemize}
\item download sources of binutils, gcc, newlib and gdb
\item compile them in place with the right options
\end{itemize}

Give a look at the script code before execution (safe behaviour), change GCC_BINARY_VERSION to 4.2 if you are under Ubuntu 8.10, and then run it:
\begin{verbatim}
$ sh ./build_am_toolchain.sh
\end{verbatim}

The arm-elf-gcc is now in your path but if you start a new terminal, don't forget to add it typing : 
\begin{verbatim}
export PATH=[the_path_to_your_gnuarm_directory]/bin:$PATH
\end{verbatim}

You can now compile an application for ARM as described in Annex \ref{compileanapplication}.


\subsection{Windows}
\begin{itemize}
\item Download a GCC-4.0.2 tool chain (\href{http://www.gnuarm.com/bu-2.16.1_gcc-4.0.2-c-c++_nl-1.14.0_gi-6.4.exe}{http://www.gnuarm.com/bu-2.16.1_gcc-4.0.2-c-c++_nl-1.14.0_gi-6.4.exe}).
\item Install only selected components in the below picture. ARM7(ATMEL AT91SAM7S256) in the NXT is Little Endian and does not have FPU.
\item Do not check "Install Cygwin DLLs..." because Cygwin was already installed.\\
\item At the end of the installation, you were asked about adding the tool path to Windows Environment Variables, but it would not be needed.
\end{itemize}

%%%%%%%%%%%%%%%%%%%%%%%%%
\section{Nexttool + Lego Drivers} \label{drivers}
%%%%%%
\subsection{MAC OS}
Download and install the Lego Drivers (\href{http://mindstorms.lego.com/en-us/support/files/default.aspx#Driver}{http://mindstorms.lego.com/en-us/support/files/default.aspx
\#Driver}) and the firmware update (\href{http://mindstorms.lego.com/en-us/support/files/default.aspx#Firmware}{http://mindstorms.lego.com/en-us/support/files/default.aspx\#Firmware}) for MAC OS. \\
Download Nexttool (\href{http://bricxcc.sourceforge.net/utilities.html}{http://bricxcc.sourceforge.net/utilities.html}) and a new firmware (\href{http://bricxcc.sourceforge.net/lms_arm_jch.zip}{http://bricxcc.sourceforge.net/lms\_arm\_jch.zip}) and update the firm-ware as explained below :
\begin{itemize}
\item Reset the NXT : To go into firmware update mode, press the reset button (at the back of the NXT, upper left corner beneath the USB connector) for more than 5 seconds while the NXT is turned on. The NXT will audibly tick when it is in firmware update mode.
\item Copy an Enhanced NXT firmware (i.e. lms\_arm\_nbcnxc\_107.rfw) to NeXTTool extracted directory.
\item Launch Nexttool, and updload the Enhanced NXT firmware to the NXT (clicking on "Download firmware"), selecting it.
	\begin{figure}[h] %  figure placement: here, top, bottom, or page
   		\centering
		\includegraphics[width=0.7\textwidth]{pictures/firmware.jpg}
	\end{figure}
\item Remove the battery from the NXT and insert it again, and then press orange rectangle button on the NXT to turn on the Enhanced NXT firmware. The Enhanced NXT firmware has same GUI as the LEGO standard firmware.
\end{itemize}
To upload a program into the NXT, go to the Annex \ref{uploadprogram}.

%%%%
\subsection{Linux}
Note: Nextool binary for Linux seems to fail for firmware upload.\\
Install required packages:
\begin{verbatim}
$sudo apt-get install scons libusb-dev libusb-0.1-4
\end{verbatim}
Download the libnxt (\href{http://libnxt.googlecode.com/files/libnxt-0.3.tar.gz}{http://libnxt.googlecode.com/files/libnxt-0.3.tar.gz}) archive and extract it. Go in the new directory and build the project with scons:
\begin{verbatim}
$ cd libnxt-0.3/ 
$ scons
\end{verbatim}

A program call fwflash is created.\\
Download John Hansen's Enhanced NXT firmware (\href{http://bricxcc.sourceforge.net/lms_arm_jch.zip}{http://bricxcc.sourceforge.net/lms\_arm\_jch.zip}) (any version numbered 106 or later includes the native-invocation feature) and store the Enhanced NXT firmware (i.e lms\_arm\_nbcnxc\_1xx.rfw) in the directory where fwflash is stored.\\
Connect the NXT brick to usb and turn it on. Then press the reset button for more than 4s to put it in firmware upload mode (nxt display is cleared but it makes a ticking sound).\\
Flash the firmware with the following command (it takes some dozen of seconds), where 1xx is replaced by the number of the firmware:
\begin{verbatim}
$ sudo ./fwflash lms_arm_nbcnxc_1xx.rfw
\end{verbatim}

Troubleshooting: After completion of the upload, sometimes NXT display is messed and block : reboot it by a quick push on the reset button or remove the battery. If the NXT makes a ticking sound, it is still in firmware upload mode. If troubles, use and see windows firmware update procedure (and LEGO UserGuide).\\

To upload a program into the NXT, go to the Annex \ref{uploadprogram}.

%%%%
\subsection{Windows}
Download and install the Lego Drivers (\href{http://mindstorms.lego.com/en-us/support/files/default.aspx#Driver}{http://mindstorms.lego.com/en-us/support/files/default.aspx
\#Driver}) for PC. \\
Download Nexttool (\href{http://bricxcc.sourceforge.net/nexttool.zip}{http://bricxcc.sourceforge.net/nexttool.zip}) and a new firmware (\href{http://bricxcc.sourceforge.net/lms_arm_jch.zip}{http://bricxcc.sourceforge.net/lms\_arm\_jch.zip}) and update the firm-ware as explained below :
\begin{itemize}
\item Reset the NXT : To go into firmware update mode, press the reset button (at the back of the NXT, upper left corner beneath the USB connector) for more than 5 seconds while the NXT is turned on. The NXT will audibly tick when it is in firmware update mode.
\item Copy an Enhanced NXT firmware (i.e. lms\_arm\_nbcnxc\_107.rfw) to NeXTTool extracted directory.
\item Execute Cygwin and type the following command to change the current directory to the NexTTool extracted directory. (NeXTTool is assumed to be extracted under C:$\backslash$cygwin$\backslash$nexttool directory)
	\begin{verbatim}
	$cd C:\cygwin\nexttool
	\end{verbatim}
\item Connect PC and the NXT by USB cable.
\item Type the following command in Cygwin to upload the Enhanced NXT firmware to the NXT (Program upload may take around half minutes and then, NXT LCD is turned to display some chunk from blank).
	\begin{verbatim}
	$./NeXTTool.exe /COM=usb -firmware=lms\_arm\_nbcnxc\_107.rfw
	\end{verbatim}
\item Remove the battery from the NXT and insert it again, and then press orange rectangle button on the NXT to turn on the Enhanced NXT firmware. The Enhanced NXT firmware has same GUI as the LEGO standard firmware.
\end{itemize}
You can now upload a program into the NXT as described in the Annex \ref{uploadprogram}.

\newpage
\chapter{Appendix}
%%%%%%%%%%%%%%%%%%%%%%%%%
\section{Launching Trampoline tests} \label{tests}
To launch the tests you have to compile ViPER \footnote{Virtual Processor Emulator, ViPER is used on Posix system to send interrupts to Trampoline to emulate the timers. It is launched by Trampoline.} first. Go in viper/ and type in a terminal :
	\begin{verbatim}
	$make
	\end{verbatim}
To launch the tests, go in check/ and type in a terminal :
	\begin{verbatim}
	$./tests.sh
	\end{verbatim}
At the end of the tests you should see :
	\begin{verbatim}
	...
	Compare results with the expected ones...
	Functional tests Succeed!!
	GOIL tests Succeed!!
	\end{verbatim}
If an error occurs, you can visit Trampoline's forum (\href{http://trampoline.rts-software.org/bb/}{http://trampoline.rts-software.org/bb/}).


%%%%%%%%%%%%%%%%%%%%%%%%%
\section{Cross-Compile an application} \label{compileanapplication}
To cross-compile a Trampoline application for ARM ports, you need to set  :
\begin{verbatim}
COMPILER = "arm-elf-gcc";
ASSEMBLER = "arm-elf-as";
LINKER = "arm-elf-ld";
\end{verbatim}
in your oil file as you can see in examples/arm/nxt/nxt\_simple.oil.\\
You also need to add the path to your libgcc and libc as below (X.X.X is your cross-gcc version) :
\begin{verbatim}
LDFLAGS = "-L[GNUARM_PATH]/lib/gcc/arm-elf/X.X.X -lgcc";
LDFLAGS = "-L[GNUARM_PATH]/arm-elf/lib -lc";
\end{verbatim}



And then, compile your application typing in a terminal (from the example in examples/arm/nxt/simple) :
	\begin{verbatim}
	$goil -t=arm/nxt --templates=../../../../goil/templates -g -i nxt_simple.oil
	\end{verbatim}
This will generate the Makefile needed, thus type :
	\begin{verbatim}
	$make
	\end{verbatim}
Then you need to upload the nxt\_simple\_exe.rxe file (see Annex \ref{uploadprogram}- after installing drivers and softwares on your platform (\ref{drivers})

%%%%%%%%%%%%%%%%%%%%%%%%%
\section{Upload a program} \label{uploadprogram}
\subsection{MAC OS}
To upload a program in the NXT (the nxt example examples/arm/nxt/simple/nxt\_simple\_exe.rxe)
\begin{itemize}
\item Connect the PC and the NXT by USB cable.
\item Launch Nexttool, select "usb port".
	\begin{figure}[htbp] %  figure placement: here, top, bottom, or page
   		\centering
		\includegraphics[width=0.25\textwidth]{pictures/usbport.jpg}
	\end{figure}
\item Go to "NXT Explorer"
	\begin{figure}[htbp] %  figure placement: here, top, bottom, or page
   		\centering
		\includegraphics[width=.7\textwidth]{pictures/nxtexplorer.jpg}
	\end{figure}
\item Click on the "Download selected files to the NXT" and select the nxt\_simple\_exe.rxe file.
\item If program upload was succeeded, you can see the nxt\_simple\_exe.rxe file in the files list as below.
	\begin{center}[h] %  figure placement: here, top, bottom, or page
   		%\centering
		\includegraphics[width=1\textwidth]{pictures/downloadfile.jpg}
	\end{center}
\item To execute a program on the NXT, go in "My files"/"Software files".
\end{itemize}

%%%%
\subsection{Linux}
Required Packages: libusb-0.1-4\\
Download this executable of John Hansen's NeXTTool (\href{http://bricxcc.sourceforge.net/nexttool.zip}{http://bricxcc.sourceforge.net/nexttool.zip}) (built from bricxcc svn repository, revision 1)\\
Check the version of NeXTTool, it should be 1.0.1.0:
\begin{verbatim}
$ sudo ./[NEXTTOOL_PATH]/NeXTTool
nexttool version 1.0.1.0 (1.0.1.0)
Copyright (c) 2006, John Hansen
Use "NeXTTool -help" for more information.
\end{verbatim}

To ubload over usb, turn on the NXT, connect it to USB and run the following command (example: nxt\_simple\_exe.rxe) :
\begin{verbatim}
$ sudo ./[NEXTTOOL_PATH]/NeXTTool /COM=usb -download=nxt\_simple\_exe.rxe
\end{verbatim}

Troubleshooting: Test the following command to see if Nexttool is working (set execution right for NeXTTool):
\begin{verbatim}
$ sudo ./[NEXTTOOL_PATH]/NeXTTool /COM=usb -versions
Protocol version = 1.124
Firmware version = 1.xx
\end{verbatim}

To ubload over bluetooth, you need to define an alias name in a file 'nxt.dat' as explained in this post: Minsdtorm 2.0 development on linux . Then turn on the NXT and run the following command (example: nxt\_simple\_exe.rxe) :
\begin{verbatim}
$ sudo ./[NEXTTOOL_PATH]/NeXTTool /COM=alias_bt -download=nxt\_simple\_exe.rxe
\end{verbatim}

To execute a program on the NXT, go in "My files"/"Software files".

%%%%
\subsection{Windows}
To upload a program in the NXT (the nxt example examples/arm/nxt/lonely\_exe.rxe) follow the steps below :
\begin{itemize}
\item Connect the PC and the NXT by USB cable.
\item Type the following command in Cygwin (from examples/arm/nxt) :
	\begin{verbatim}
	$./[NEXTTOOL_PATH]/NeXTTool.exe /COM=usb -download=lonely_exe.rxe
	$./[NEXTTOOL_PATH]/NeXTTool.exe /COM=usb -listfiles=lonely_exe.rxe
	\end{verbatim}
\item If program upload was succeeded, program size could be displayed in Cygwin such as the second line in the below command outputs. 
	\begin{verbatim}
	Executing NeXTTool to upload helloworld.rxe...
	helloworld.rxe=15280
	NeXTTool is terminated.
	\end{verbatim}
\item To execute a program on the NXT, go in "My files"/"Software files".
\end{itemize}

%\end{document}

