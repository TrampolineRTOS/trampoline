
\subsubsection{Olimex LPC-E2294}

%%%%%%%%%%%%%%%%%%%%%%%%%
\paragraph{OPENOCD + drivers}
%%%%%%
openOCD is a software that communicates via the JTAG (Open On-Chip Debug Solution for Embedded Target Systems based on the ARM7 and ARM9 Family). It's an open source software by BerliOS (http://openocd.berlios.de/web/). \\
Download and compile openOCD for your achitecture (it appears revision after 1200 are not able to compile correcty under MacOS X).\\
As you don't need just openOCD because it needs the USB driver, you've got two choice : 
\begin{itemize}
\item install libusb et libftdi
\item install D2XX
\end{itemize}

%%%%%%%%%%%%%%%%%%%%%%%%%
\paragraph{Upload a program}
To upload a program double click on \textit{2-run-openocd.command} to open the connection between your computer and the board.\\
And double click on \textit{2a-debug-external-ram.command} if you want to upload your program in external ram with debug.\\
In the debug consol type :
\begin{itemize}
\item "c" to continue
\item "b" to place a breakpoint
\item "si" to step instruction
\item "display/i \$pc" to display assembler code on the pc register
\item "x/i \$pc" to visualise the i blocs from pc
\item "info registers" to see the registers state
\end{itemize}

