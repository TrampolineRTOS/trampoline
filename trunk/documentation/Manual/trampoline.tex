\documentclass[11pt]{manual}
\usepackage[utf8]{inputenc}
\usepackage[a4paper,footnotesep=15pt,foot=40pt]{geometry}                % See geometry.pdf to learn the layout options. There are lots.
%\usepackage[parfill]{parskip}    % Activate to begin paragraphs with an empty line rather than an indent
\usepackage{graphicx}
\usepackage{listings}
\usepackage{color}
%\usepackage{amssymb}
\usepackage{makeidx}
\usepackage{url}
\usepackage{longtable}
\usepackage{colortbl}
\usepackage[table]{xcolor}
\usepackage{underscore}
\usepackage{caption}
\usepackage{amsmath}
\usepackage{datatool}

\renewcommand{\captionlabelfont}{\small \bf}
\renewcommand{\captionfont}{\small \it}

\definecolor{seagreen}{RGB}{64,100,120}
\definecolor{darkpink}{RGB}{200,100,120}
\definecolor{light-gray}{gray}{0.85}
\definecolor{medium-gray}{gray}{0.5}
\usepackage[colorlinks=true,linkcolor=seagreen,citecolor=darkpink]{hyperref}
%\usepackage{supertabular} %supertabular

\usepackage{float} %figures

%
% Commands for paragraph types
%
%\newcommand{\note}[1]{\par\begingroup\hangindent=2em\hangafter=1 Note: #1\par\endgroup}

%
% Commands for inline formatting of C computing language
%
%\newcommand{\ctype}[1]{{\ttfamily #1}}
%\newcommand{\cdata}[1]{{\ttfamily #1}}
%\newcommand{\var}[1]{{\ttfamily #1}}
%\newcommand{\cfunction}[1]{{\ttfamily #1}}
\newcommand{\cmacro}[1]{{\ttfamily #1}}
%\newcommand{\constant}[1]{{\ttfamily #1}}

%
% Commands for other formatting
%
%\newcommand{\file}[1]{\ttfamily `#1'}

\newcommand{\ie}{i.e.}
\newcommand{\asfct}[1]{\texttt{#1}}
\newcommand{\hex}[1]{\texttt{#1$_H$}}
\newcommand{\reg}[1]{\textit{#1}}
\newcommand{\idxconfflag}[1]{#1\index{Configuration macros!#1}\index{#1}}
\newcommand{\toreplace}[1]{{\footnotesize $<$}#1{\footnotesize $>$}}
\newcommand{\figcaption}[2]{\textit{\caption{\small #1}}\label{#2}}
\newcommand{\goil}{goil\index{goil, The oil compiler of Trampoline}}
\newcommand{\process}{process\index{process, a task, basic or extended, or an ISR category 2}}
\newcommand{\sysgen}{system generation tool\index{system generation tool, the tool that takes as an input a description of the system (in OIL or in XML) to generate the corresponding .c and .h files.}}
\newcommand{\stringlit}[1]{``{#1}''}
\newcommand{\api}[1]{\cfunction{#1}\index{#1}}
\newcommand{\autosar}{AUTOSAR}
\newenvironment{penum}{
\begin{enumerate}
  \setlength{\itemsep}{1pt}
  \setlength{\parskip}{0pt}
  \setlength{\parsep}{0pt}
}{\end{enumerate}}

%
% Graphic symbols for scheduling diagrams
%
\newcommand{\activate}{\includegraphics[height=9pt]{pictures/activate.pdf}}
\newcommand{\terminate}{\includegraphics[height=9pt]{pictures/terminate.pdf}}

%
% states of a task
%
\newcommand{\RUNNING}{\constant{RUNNING}\index{RUNNING, state of a task}}
\newcommand{\READY}{\constant{READY}\index{READY, state of a task}}
\newcommand{\SUSPENDED}{\constant{SUSPENDED}\index{SUSPENDED, state of a task}}
\newcommand{\WAITING}{\constant{WAITING}\index{WAITING, state of a task}}
\newcommand{\AUTOSTART}{\constant{AUTOSTART}\index{AUTOSTART, state of a task}}
\newcommand{\READYANDNEW}{\constant{READY_AND_NEW}\index{READY_AND_NEW, state of a task}}

%
% result codes
%
\newcommand{\OK}{E\_OK}
\newcommand{\OSID}{E\_OS\_ID}
\newcommand{\OSLIMIT}{E\_OS\_LIMIT}
\newcommand{\OSRESOURCE}{E\_OS\_RESOURCE}
\newcommand{\OSCALLLEVEL}{E\_OS\_CALLEVEL}

%
% OIL attribute
%
\newcommand{\PRIORITY}{PRIORITY\index{PRIORITY}}
\newcommand{\SCHEDULE}{SCHEDULE\index{SCHEDULE}}

\newcommand{\servicename}{}
\newcommand{\servicedescription}{}
\newcommand{\servicereturn}{}
\DTLnewdb{argDB}
\DTLnewdb{resDB}
\newenvironment{service}[2]{%
  \subsection{#1}\index{#1}%
  \renewcommand{\servicename}{#1}%
  \renewcommand{\servicereturn}{#2}
  \DTLcleardb{argDB}
  \DTLcleardb{resDB}
}{
 
  {\bf Prototype of \servicename\ is:}
 
  {\tt \servicereturn\ \servicename(%
  \DTLforeach*{argDB}{\argumenttype=type,\argumentname=name}{%
    \DTLiffirstrow{}{, }\argumenttype\ \argumentname%
  }%
  );}

  \DTLifdbempty{argDB}{}{
    {\bf Arguments of \servicename\ are:}
  
    \DTLforeach*{argDB}{\argumenttype=type,\argumentname=name,\argumentdesc=desc}{%
      \par\begingroup\hangindent=2em\hangafter=1 {\ttfamily\argumentname}\hspace{1em}\argumentdesc.\par\endgroup
    }
  }
  
  \DTLifdbempty{resDB}{}{
    {\bf Status codes returned by \servicename:}
  
    \DTLforeach*{resDB}{\statusres=value,\statusdesc=desc,\statusstatus=status}{%
      \par\begingroup\hangindent=2em\hangafter=1 {\ttfamily\statusres}\hspace{1em}\statusdesc\ (\statusstatus).\par\endgroup
    }
  }
}

\newcommand{\argument}[3]{%
  \DTLnewrow{argDB}
  \DTLnewdbentry{argDB}{type}{#1}
  \DTLnewdbentry{argDB}{name}{#2}
  \DTLnewdbentry{argDB}{desc}{#3}
}

\newcommand{\resultcode}[2]{%
  \DTLnewrow{resDB}
  \DTLnewdbentry{resDB}{value}{#1}
  \DTLnewdbentry{resDB}{desc}{#2}
  \DTLnewdbentry{resDB}{status}{extended and standard}
}
  
\newcommand{\resultcodeext}[2]{%
  \DTLnewrow{resDB}
  \DTLnewdbentry{resDB}{value}{#1}
  \DTLnewdbentry{resDB}{desc}{#2}
  \DTLnewdbentry{resDB}{status}{extended only}
}


% Tables.
%
%\newenvironment{tableii}[4]{%
%  \begin{center}%
%    \def\lineii##1##2{\csname#2\endcsname{##1}&##2\\}%
%    \begin{tabular}{#1}{\bfseries #3}&{\bfseries #4} \\* \hline%
%}{%
%    \end{tabular}%
%  \end{center}%
%}
%
%\newenvironment{longtableii}[4]{%
%  \begin{center}%
%    \def\lineii##1##2{\csname#2\endcsname{##1}&##2\\}%
%    \begin{longtable}[c]{#1}{\bfseries #3}&{\bfseries #4} \\* \hline\endhead%
%}{%
%    \end{longtable}%
%  \end{center}%
%}
%
%\newenvironment{tableiii}[5]{%
%  \begin{center}%
%    \def\lineiii##1##2##3{\csname#2\endcsname{##1}&##2&##3\\}%
%    \begin{tabular}{#1}{\bfseries #3}&{\bfseries #4}&{\bfseries #5} \\%
%      \hline%
%}{%
%    \end{tabular}%
%  \end{center}%
%}
%
%\newenvironment{longtableiii}[5]{%
%  \begin{center}%
%    \def\lineiii##1##2##3{\csname#2\endcsname{##1}&##2&##3\\}%
%    \begin{longtable}[c]{#1}{\bfseries #3}&{\bfseries #4}&{\bfseries #5} \\%
%      \hline\endhead%
%}{%
%    \end{longtable}%
%  \end{center}%
%}
%
%\newenvironment{tableiv}[6]{%
%  \begin{center}%
%    \def\lineiv##1##2##3##4{\csname#2\endcsname{##1}&##2&##3&##4\\}%
%    \begin{tabular}{#1}{\bfseries #3}&{\bfseries #4}&{\bfseries #5}&{\bfseries #6} \\%
%      \hline%
%}{%
%    \end{tabular}%
%  \end{center}%
%}
%
%\newenvironment{longtableiv}[6]{%
%  \begin{center}%
%    \def\lineiv##1##2##3##4{\csname#2\endcsname{##1}&##2&##3&##4\\}%
%    \begin{longtable}[c]{#1}{\bfseries #3}&{\bfseries #4}&{\bfseries #5}&{\bfseries #6}%
%      \\%
%      \hline\endhead%
%}{%
%    \end{longtable}%
%  \end{center}%
%}
%
%\newenvironment{tablev}[7]{%
%  \begin{center}%
%    \def\linev##1##2##3##4##5{\csname#2\endcsname{##1}&##2&##3&##4&##5\\}%
%    \begin{tabular}{#1}{\bfseries #3}&{\bfseries #4}&{\bfseries #5}&{\bfseries #6}&{\bfseries #7} \\%
%      \hline%
%}{%
%    \end{tabular}%
%  \end{center}%
%}
%
%\newenvironment{longtablev}[7]{%
%  \begin{center}%
%    \def\linev##1##2##3##4##5{\csname#2\endcsname{##1}&##2&##3&##4&##5\\}%
%    \begin{longtable}[c]{#1}{\bfseries #3}&{\bfseries #4}&{\bfseries #5}&{\bfseries #6}&{\bfseries #7}%
%      \\%
%      \hline\endhead%
%}{%
%    \end{longtable}%
%  \end{center}%
%}

\lstdefinelanguage{goilTemplate}
{
  morekeywords= {
	  after,
	  before,
	  between,
	  block,
	  default,
	  do,
	  else,
	  elsif,
	  emptylist,
	  end,
	  error,
	  exists,
	  false,
	  for,
	  foreach,
	  from,
	  here,
	  let,
	  loop,
	  no,
	  if,
	  in,
	  mod,
	  not,
	  or,
	  prefixedby,
	  template,
	  then,
	  to,
	  true,
	  yes,
	  warning,
	  while,
	  write,
	  !,
	  ?
	}
}

\lstset{
  language=goilTemplate,
  emph={
    var,
    expression,
    string,
    instruction_list,
    template_file_name,
    reader,
    hierarchy
  },
  emphstyle=\em,
  moredelim=[s][\color{blue}]{\%}{\%},
  moredelim=[s]{"}{"},
  morecomment=[l][\color{medium-gray}\itshape]{\#},
  basicstyle=\ttfamily\small
}


%\lstset{frame=leftline}

\title{The Trampoline handbook}
\author{Jean-Luc B\'echennec\\Florent Pavin\\Pierre Molinaro}

%\release{2.0}
\makeindex

\begin{document}
\maketitle
~\newpage
\setcounter{tocdepth}{2}
\tableofcontents

\part{The Real-Time Operating System}

Each chapter in this part starts with a presentation of the API and ends with implementation details related to the subject.

%!TEX root = ./main.tex

\chapter{Tasks}
\label{chap:tasks}

\lettrine{A} task is an execution framework for the functions of the application\,\footnote{The term {\em Application} is also used in \autosar\ to designate a set of object, this manual uses OS Application to name the \autosar\ applications and Application to name the user level software.}. A task is a kind of \process. Tasks are executed concurrently and asynchronously, see \ref{sec:scheduling}. 2 kinds of task exist: basic tasks and extended tasks. A basic task cannot block (\ie\ it cannot use a service that may block) while an extended task can.
The tasks and their properties are declared in the OIL file, see \ref{oil:task}. Their functions are defined in a C file.

\section{States a task}
\label{sec:taskstate}

A task may be in different states. A basic task may be currently executing (in the \RUNNING\ state), ready to execute (in the \READY\ state) or not active at all (in the \SUSPENDED\ state). Figure \ref{fig:basictaskstates} shows the states of a basic task. An extended task has an additional \WAITING\ state.  Figure \ref{fig:extendedtaskstates} shows the states of an extended task. See section \ref{sec:internaltaskstate} for additional informations about the states of a task.

\begin{figure}[htbp] %  figure placement: here, top, bottom, or page
   \centering
   \parbox{.45\linewidth}{%
     \includegraphics[scale=.6]{pictures/statesBasic.pdf} 
     \caption{States of a \BASIC\ task.}%
     \label{fig:basictaskstates}}%
%   \qquad
   \begin{minipage}{.55\linewidth}%
     \includegraphics[scale=.6]{pictures/statesExtended.pdf} 
     \caption{States of an \EXTENDED\ task.}%
     \label{fig:extendedtaskstates}%
   \end{minipage}%
\end{figure} 


A task goes from one state to the other according to various conditions as shown in table \ref{tab:statetrans}.

\begin{table}[htbp]
\caption{Transition from state to state of a task.}
\rowcolors{1}{white}{light-gray}
\begin{longtable}[c]{l|l|l|p{7cm}}
\bf transition & \bf former state & \bf new state & \bf description\\
\hline
activate & \SUSPENDED & \READY & the task is set in the \READY\ state on one of the following occurrences: services \api{ActivateTask} or \api{ChainTask}, activation notification coming from an alarm, a schedule table or a message. \\
start & \READY & \RUNNING & the task is set to the running state and begin to execute because it has the highest priority in the system and has been elected by the scheduler. \\
terminate & \RUNNING & \SUSPENDED & the task is set to the \SUSPENDED\ state when it calls the \api{TerminateTask} or \api{ChainTask} service.\\
preempt & \RUNNING & \READY & the task is set to the \READY\ state when the scheduler starts a higher priority task.\\
wait & \RUNNING & \WAITING & the task may be set to the \WAITING\ state when it calls the service \api{WaitEvent}.\\
release & \WAITING & \READY & the task is set to the \READY\ state when it gets one of the  events it is waiting for. \\
\end{longtable}
\label{tab:statetrans}
\end{table}

\note{A system service may do more than one transition at a time. For instance, if a task is activated by calling \cfunction{ActivateTask} and its priority is higher than the priority of the current running task, the new task will go from \SUSPENDED\ to \RUNNING\ and the intermediate state \READY\ will not be observable.}

\section{The scheduling}
\label{sec:scheduling}

Trampoline schedules the tasks dynamically during the execution of the application. A task is scheduled according to its priority and whether it is preemptable or not. The priority of a task is given at design stage, and indicated in the OIL file using the \PRIORITY\ attribute, see \ref{sec:oiltask}, and may change during execution when the task gets or release a resource. The preemptability of a task may be set too. It is also indicated in the OIL file using the \SCHEDULE\ attribute, see \ref{sec:oiltask}.

A tasks continues to run until it is preempted because a task having a higher priority is put in the \READY\ state,  or it blocks because it is waiting for an event. Only extended tasks may block. If more than one task have the same priority, tasks are run one after the other because a task may not preempt an other task having the same priority. So there is no round robin among tasks of the same priority level.

A non-preemptable task runs until it calls \api{Schedule} and a higher priority task is in the \READY\ state or until it blocks. More informations about priority and preemptability may be found in chapter \ref{chap:resources}.

In the following examples, the horizontal axis is the time.  The state of the task is indicated in a rectangle that spans a period of time. When the task is running the rectangle is grayed. An up arrow \activate\ indicates a task activation and a down arrow \terminate\ a task termination.

\begin{figure}[htbp] %  figure placement: here, top, bottom, or page
   \centering
   \includegraphics[scale=.7]{pictures/schedulingPreempt.pdf} 
   \caption{{\bfseries Scheduling of preemptable tasks.} During A period, $T_1$ is {\sffamily\scshape running} and $T_2$ is {\sffamily\scshape suspended}. Then $T_2$ is activated. Since $Prio(T_2) > Prio(T_1)$, $T_1$ is preempted and $T_2$ runs (B period). $T_2$ terminates and $T_1$ becomes {\sffamily\scshape running} again (C period) until it terminates.}
   \label{fig:schedulePreempt}
\end{figure} 

\begin{figure}[htbp] %  figure placement: here, top, bottom, or page
   \centering
   \includegraphics[scale=.7]{pictures/schedulingNonPreempt.pdf} 
   \caption{{\bfseries Scheduling of non-preemptable tasks.} During A period, $T_1$ is {\sffamily\scshape running} and $T_2$ is {\sffamily\scshape suspended}. Then $T_2$ is activated. Even if $Prio(T_2) > Prio(T_1)$, $T_1$ is non-preemptable and continues to run until it terminates (B period). In the meantime, $T_2$ is {\sffamily\scshape ready}. $T_1$ terminates and $T_2$ runs (C period) until it terminates.}
   \label{fig:schedulePreempt}
\end{figure} 

\section{Writing the code of a task}

Trampoline provides a \cmacro{TASK} macro to define a task in a C source file. The macro takes one argument which is the identifier of the task:

%\newpage
\begin{lstlisting}[language=C]
TASK(MyTask)
{
  /* code of the task */
  
  TerminateTask();
}
\end{lstlisting}

The code of the task is plain C.

The task should always end with a call to the \api{TerminateTask} service. See \ref{api:TerminateTask}.


\section{Tasks services}

\begin{service}{DeclareTask}{}

Each task has an identifier of type \ctype{TaskType}. This identifier is declared in the OIL file and is used in system calls to refer to a particular task. Before using such an identifier in your program, you have to declare it:

\begin{lstlisting}[language=C]
DeclareTask(MyTask);
\end{lstlisting}

This makes the \cdata{MyTask} identifier available in the current scope.

\note{ \servicename\ is a C macro. When the task has been define above using the macro \cmacro{TASK}, the identifier of the task is already in the scope and \cmacro{DeclareTask} is not needed.}

\argument{TaskType}{TaskID}{The id of the task to declare}
\end{service}

\begin{service}{ActivateTask}{StatusType}

\note{This service does a rescheduling}

Activates a new instance of a task. If activation counter has reached the maximum activation count or the task cannot be activated for timing protection purpose, the service fails. Otherwise if an instance is already active (\RUNNING\ or \READY), the state does not change and the activation is recorded to be done later. If no instance is active, the state of the task is changed to \READY.

Figures \ref{fig:scheduleT1lp}, \ref{fig:scheduleT1hp} and \ref{fig:scheduleMultiple} show 2 examples of task activation.

\argument{TaskType}{TaskID}{The id of the task to activate}
\resultcode{\OK}{No error, the task has been successfully activated}
\resultcodeext{\OSID}{Invalid TaskID. No task with such an id exists}
\resultcode{\OSLIMIT}{Too many activations of the task}
\end{service}

\begin{figure}[htbp] %  figure placement: here, top, bottom, or page
   \centering
   \begin{minipage}[c]{.6\linewidth}
   \includegraphics[scale=.7]{pictures/schedulingT1lp.pdf}
   \end{minipage}\hfill
   \begin{minipage}[c]{.3\linewidth}
   \begin{lstlisting}[language=C]
   TASK(T2) {
     ... /* C period */
     TerminateTask();
   }
   
   TASK(T1) {
     ... /* A period */
     ActivateTask(T2);
     ... /* B period */
     TerminateTask();
   }
 \end{lstlisting}
   \end{minipage}
   \caption{{\bfseries Activation of a lower priority task.} $Prio(T_1) \ge Prio(T_2)$. During A period, $T_1$ is {\sffamily\scshape running} and $T_2$ is {\sffamily\scshape suspended}. Then $T_1$ calls {\upshape\ttfamily ActivateTask(T2);}. Since $T_2$ does not have a higher priority, it becomes {\sffamily\scshape ready} (B period).  $T_1$ terminates and $T_2$ runs (C period) until it terminates.}
   \label{fig:scheduleT1lp}
\end{figure} 

\begin{figure}[htbp] %  figure placement: here, top, bottom, or page
   \centering
   \begin{minipage}[c]{.6\linewidth}
   \includegraphics[scale=.7]{pictures/schedulingT1hp.pdf}
   \end{minipage}\hfill
   \begin{minipage}[c]{.3\linewidth}
   \begin{lstlisting}[language=C]
   TASK(T2) {
     ... /* B period */
     TerminateTask();
   }
   
   TASK(T1) {
     ... /* A period */
     ActivateTask(T2);
     ... /* C period */
     TerminateTask();
   }
 \end{lstlisting}
   \end{minipage}
   \caption{{\bfseries Activation of a higher priority task.} $Prio(T_1) < Prio(T_2)$. During A period, $T_1$ is {\sffamily\scshape running} and $T_2$ is {\sffamily\scshape suspended}. Then $T_1$ calls {\upshape\ttfamily ActivateTask(T2);}. Since $T_2$ has a higher priority, it becomes {\sffamily\scshape running} (B period).  $T_2$ terminates and $T_1$ resumes (C period) until it terminates.}
   \label{fig:scheduleT1hp}
\end{figure} 


\begin{figure}[htbp] %  figure placement: here, top, bottom, or page
   \centering
   \begin{minipage}[c]{.6\linewidth}
   \includegraphics[scale=.7]{pictures/schedulingMultiple.pdf}
   \end{minipage}\hfill
   \begin{minipage}[c]{.3\linewidth}
   \begin{lstlisting}[language=C]
   TASK(T2) {
     ... /* C period */
     TerminateTask();
   }
   
   TASK(T1) {
     ... /* A period */
     ActivateTask(T2);
     ActivateTask(T2);
     ActivateTask(T2);
     ... /* B period */
     TerminateTask();
   }
 \end{lstlisting}
   \end{minipage}
   \caption{{\bfseries Multiple activations of a lower priority task.} $Prio(T_1) \ge Prio(T_2)$. During A period, $T_1$ is {\sffamily\scshape running} and $T_2$ is {\sffamily\scshape suspended}. Then $T_1$ calls {\upshape\ttfamily ActivateTask(T2);} 3 times. Since $T_1$ has a higher priority, $T_2$ does not run immediately and the 3 activations are recorded provided the ACTIVATION attribute in the OIL description of the task is a least 3 (B period). When $T_1$ terminates, the scheduler executes $T_2$ 3 times (C periods).}
   \label{fig:scheduleMultiple}
\end{figure} 



\begin{service}{ChainTask}{StatusType}

\note{This service does a rescheduling}

This service puts task TaskID in \READY\ state, and the calling task in the \SUSPENDED\ state. It acts as the \api{TerminateTask} service for the calling task.
\argument{TaskType}{TaskID}{The id of the task to activate}
\resultcode{\OK}{No error, the task TaskID has been successfully activated and the calling task has been successfully terminated. Note in this case {\ttfamily\servicename} does not return so actually \OK\ is never returned}
\resultcodeext{\OSID}{Invalid TaskID. No task with such an id exists}
\resultcode{\OSLIMIT}{Too many activations of the task}
\resultcodeext{\OSRESOURCE}{The calling task still held a resource}
\resultcodeext{\OSCALLLEVEL}{Called outside of a task}
\end{service}

\begin{figure}[htbp] %  figure placement: here, top, bottom, or page
   \centering
   \begin{minipage}[c]{.6\linewidth}
   \includegraphics[scale=.7]{pictures/schedulingChain.pdf}
   \end{minipage}\hfill
   \begin{minipage}[c]{.3\linewidth}
   \begin{lstlisting}[language=C]
   TASK(T2) {
     ... /* B period */
     TerminateTask();
   }
   
   TASK(T1) {
     ... /* A period */
     ChainTask(T2);
   }
 \end{lstlisting}
   \end{minipage}
   \caption{{\bfseries Chaining of tasks.} During A period, $T_1$ is {\sffamily\scshape running} and $T_2$ is {\sffamily\scshape suspended}. Then $T_1$ calls {\upshape\ttfamily ChainTask(T2);}. $T_1$ terminates and $T_2$ is activated. Then $T_2$ runs (B periods).}
   \label{fig:scheduleMultiple}
\end{figure} 

\begin{service}{TerminateTask}{StatusType}

\note{This service does a rescheduling}

This service stops the calling task and puts it in \SUSPENDED\ state.
\resultcode{\OK}{No error, the calling task has been successfully terminated. Note in this case {\ttfamily\servicename} does not return so actually \OK\ is never returned}
\resultcodeext{\OSRESOURCE}{The calling task still held a resource}
\resultcodeext{\OSCALLLEVEL}{Called outside of a task}
\end{service}

\begin{service}{Schedule}{StatusType}
\note{This service does a rescheduling. 
Schedule does not deal directly with tasks but since it is a call to the scheduler, it is presented here.}

If called from a preemptable task that does not use an internal resource, Schedule has not effect. If called from a preemptable or a task that uses an internal resource, the priority of the task revert to its base priority and a rescheduling occurs.

Schedule allows to implement cooperative multitasking to insure synchronous rescheduling.
\resultcode{\OK}{No error.}
\resultcodeext{\OSRESOURCE}{The calling task still held a resource}
\resultcodeext{\OSCALLLEVEL}{Called outside of a task}
\end{service}

\begin{service}{GetTaskID}{StatusType}
\cfunction{\servicename} writes in the \var{TaskID} variable passed as reference the identifier of the task currently \RUNNING. If no task is currently \RUNNING\ because \cfunction{\servicename} was called from an ISR of before Trampoline is started, \constant{INVALID_TASK} is got.

\warning{The argument is a pointer. Do not pass an uninitialized pointer. Proper use of this service supposes a \ctype{TaskType} variable is instantiated, then its address is passed to \cfunction{\servicename} as shown in the example below:}

\begin{lstlisting}[language=C]
TaskType runningTaskID;
GetTaskID(&runningTaskID);
\end{lstlisting}

\resultcode{\OK}{No error.}
\resultcodeMP{\OSPROTECTIONMEMORY}{The caller does not have access to the addresses of \var{TaskID} reference} 
\argument{TaskRefType}{TaskID}{Reference to the task}
\end{service}

\begin{service}{GetTaskState}{StatusType}
\cfunction{\servicename} writes in the variable passed as reference in \var{State} the state of the task given in \var{TaskID}.

\warning{The \var{State} argument is a pointer. Do not pass an uninitialized pointer. Proper use of this service supposes a \ctype{TaskState} variable is instantiated, then its address is passed to \cfunction{\servicename} as shown in the example below:}

\begin{lstlisting}[language=C]
TaskStateType T1State;
GetTaskState(T1, &T1State);
\end{lstlisting}

\resultcode{\OK}{No error.}
\resultcodeext{\OSID}{Invalid TaskID. No task with such an id exists}
\resultcodeMP{\OSPROTECTIONMEMORY}{The caller does not have access to the addresses of \var{State} reference} 
\argument{TaskType}{TaskID}{The id of the task.}
\argument{TaskStateRefType}{State}{Reference to the state.}
\end{service}


%\strong{Processes} are both Tasks and ISRs category 2. Trampoline category 2 ISRs are like Basic Tasks except the priority level of the interrupt controller is raised to the priority of the ISR while the later is running.

\section{Inside Task management}
\label{sec:addtaskstate}

\subsection{Static attributes}

A task has the following static attributes:

\begin{description}
\item[The entry point of the task.] A pointer to the code of the task. When the scheduler start a task instance the first time, it uses this pointer to begin the execution.
\item[The internal resource] the task uses if any. An internal resource is automatically taken when a task enters the \RUNNING\ state and automatically released when the task leaves the \RUNNING\ state. See \ref{sec:internalresources} for more informations.
\item[The base priority] of the task as specified in the OIL file. This priority is used to reset the current priority when the task is activated.
\item[The maximum activation count] of the task as specified in the OIL file.
\item[The kind of task,] \BASIC\ or \EXTENDED.
\item[The task id.] Used for internal checking.
\item[The id of the OS Application] the tasks belong to (only available in \autosar\ \SC{3} and \SC{4}).
\item[The timing protection configuration] if any (only available in \autosar\ \SC{2} and \SC{4}).
\end{description}

\subsection{Dynamic attributes}

A task has also the following dynamic attributes:

\begin{description}
\item[The context.] This is the chunk of RAM where the current execution context of a task is stored when the task is in the \READY\ or \WAITING\ state. The execution context is the value of the microprocessor's registers (program counter, stack pointer, other working registers). So the context depends on the target on which Trampoline runs.
\item[The stack(s).] This is the chunk of RAM where registers are pushed for function call. This attributes depends on the target architecture. For instance, the C166 micro-controller uses 2 stacks.
\item[The current activation count.] When a task is activated while not in \SUSPENDED\ state, the activation is recorded and is actually done when the task returns to the \SUSPENDED\ state.  Many activation may be recorded according to the value given to the \oilattr{ACTIVATION} task OIL attribute. When a task is activated, the current activation count is compared to the maximum activation count and if $\ge$, the activation fails.
\item[The list of resources] the task currently owns.
\item[The current priority] of the task. This priority starts equal to the basic priority and may increase when the task get a resource.
\item[The state of the task] as defined in sections \ref{sec:taskstate} and \ref{sec:internaltaskstate}.
\item[The trusted counter.] If $=0$, the task is non-trusted. If $>0$ the task is trusted. See chapter \ref{chap:osapplications} for more informations. This counter is available if Trampoline is compiled with memory protection support.
\item[The activation allowed flag.] If true, the task may be activated. If false, it cannot be activated. This flag is set by the timing protection facility. It is available if Trampoline is compiled with timing protection support. See chapter \ref{chap:timingprotection}.
\end{description}

\subsection{Additional task states}
\label{sec:internaltaskstate}

In addition to states presented in section \ref{sec:taskstate}, 2 extra states are used for internal management: 

\begin{description}
\item[\AUTOSTART] This state is used to indicate what task should be started automatically when \api{StartOS} is called. An \AUTOSTART\ task is in this initial state but no task is in this state once the application code is running. \api{StartOS} iterates through the tasks and activates those that are in the \AUTOSTART\ state.
\item[\READYANDNEW] This state is used to flag a task that is ready but has its context uninitialized. This happens when the task has just been activated. The kernel initializes the context of the task the first time it goes to the \RUNNING\ state.
\end{description}

Figure \ref{fig:states} show a complete task state automaton for both basic and extended tasks with these states added.

\begin{figure}[htbp] %  figure placement: here, top, bottom, or page
   \centering
   \includegraphics[width=4in]{pictures/states.pdf} 
   \caption{States of a task in Trampoline. \cmacro{AUTOSTART} is the initial state of autostart tasks. \cmacro{SUSPENDED} is the initial state of both non autostart tasks.}
   \label{fig:states}
\end{figure} 


\section{The {\em idle} task}

The {\em idle} task is activated by \api{StartOS}. It is a \BASIC\ task with a priority of 0 (\ie\ the lowest priority in the system, the lowest priority of tasks defined in the application is 1). So when no other task is currently running, the {\em idle} task run.

To be able to use specific platform capabilities (to put the micro-controller in stand by mode for example), this task calls repetitively a hardware specific function called \cfunction{tpl_sleep} (defined in \textit{machines/}). The tasks is then able to quantify the microprocessor occupation.

GOIL doesn't produce anything about this idle task (unlike application(s) task(s)). The idle task descriptor is defined in \file{tpl_os_kernel.c}.
%!TEX root = ./trampoline.tex

\chapter{OS Applications}

OS Applications are a set of objects managed by Trampoline and sharing common data and access rights.

\section{Execution of the OS Applications startup and shutdown hooks}

These hooks are executed from the kernel but with the access right of a task belonging to the OS Application. The \sysgen\ should choose one of the tasks of the OS Application to be used as context to execute the OS Application startup and shutdown hooks. Execution of an OS Application startup hook is done by the \cfunction{tpl_call_startup_hook_and_resume} function. The argument of this function is a function pointer to the hook. Similarly execution of an OS Application shutdown hook is done by the \cfunction{tpl_call_shutdown_hook_and_resume} function. These functions end by a call to \api{NextStartupHook} and \api{NextShutdownHook} services respectively to cycle through the hooks.

%!TEX root = ./main.tex

\chapter{Timing Protection Implementation}

The Timing Protection Implementation uses 2 timers. The first one is a {\em Free Running Timer} (FRT) which is used for {\em Time Frame}. The second one is a classical timer called {\em Timing Protection Timer} (TPT) which is used for \emph{Execution Time Budget}, \emph{Resource Locking Budget} and \emph{Interrupt Disabling Budget}.

\section{Low Level Functions}

These functions are provided by the {\em Board Support Package} and are used to manage the timers needed by the Timing Protection.

\subsection{FRT related functions}

\paragraph{\cfunction{tpl_status tpl_start_frt(void)}} starts the FRT. On a microcontroller having a FRT that starts automatically when the system is powered on, this function does nothing but must be present since it is called by Trampoline in initialization stage. An error code is returned: {\em E\_OK} means no error, {\em E\_OS\_NOFUNC} means the FRT could not be started.

\paragraph{\cfunction{tpl_status tpl_read_frt(tpl_tp_tick *out_value)}} write the current value of the FRT in \var{out_value}. An error code is returned: {\em E\_OK} means no error, {\em E\_OS\_NOFUNC} means the FRT could not be read.

\paragraph{\cfunction{tpl_status tpl_elapsed_frt(tpl_tp_tick last_tick, tpl_tp_tick *out_value)}} write the number of ticks elapsed since \var{last_tick} in \var{out_value}. If the FRT has overflown/underflown between the time \var{last_tick} was get and the time \cfunction{tpl_elapsed_frt} is called, \cfunction{tpl_elapsed_frt} gives a correct value. An error code is returned: {\em E\_OK} means no error, {\em E\_OS\_NOFUNC} means the FRT could not be read.

\subsection{TPT related functions}

\paragraph{\cfunction{tpl_status tpl_init_tpt(???)}} initializes the TPT. An error code is returned: {\em E\_OK} means no error, {\em E\_OS\_NOFUNC} means the TPT could not be initialized.

\paragraph{\cfunction{tpl_status tpl_deinit_tpt(void)}} deinitializes the TPT. An error code is returned: {\em E\_OK} means no error, {\em E\_OS\_NOFUNC} means the TPT could not be deinitialized.

\paragraph{\cfunction{tpl_status tpl_start_tpt(tpl_tp_tick delay)}} starts the TPT with an expiration delay equal to \var{delay} ticks. At that time, the \cfunction{tpl_tpt_handler} function is called. An error code is returned: {\em E\_OK} means no error, {\em E\_OS\_NOFUNC} means the TPT could not be started because it is not initialized.

\paragraph{\cfunction{tpl_status tpl_read_tpt(tpl_tp_tick *out_value)}} write the current value of the TPT in \var{out_value}. An error code is returned: {\em E\_OK} means no error, {\em E\_OS\_NOFUNC} means the TPT could not be read.

\paragraph{\cfunction{tpl_status tpl_elapsed_tpt(tpl_tp_tick last_tick, tpl_tp_tick *out_value)}} write the number of ticks elapsed since \var{last_tick} in \var{out_value}. An error code is returned: {\em E\_OK} means no error, {\em E\_OS\_NOFUNC} means the TPT could not be read.




%% HEAD SUPERTABULAR %%
\rowcolors{1}{light-gray}{white}
\newlength{\Li}\settowidth{\Li}{\textbf{Decimal}}
\newlength{\Liii}\settowidth{\Liii}{(bootstrap  )}
\newlength{\Lii}\settowidth{\Lii}{SCHEDULETABLE\_AUTOSTART }
\tablefirsthead{ \textbf{Decimal Value} & \textbf{bit 5}  & \textbf{bit 4} & \textbf{bit 3} & \textbf{bit 2} & \textbf{bit 1} & \textbf{bit 0} & \textbf{Meaning} \\ \hline }
\tablehead{ \rowcolor{white} \rowcolors{1}{light-gray}{white} \textbf{Decimal Value} & \textbf{bit 5}  & \textbf{bit 4} & \textbf{bit 3} & \textbf{bit 2} & \textbf{bit 1} & \textbf{bit 0} & \textbf{Meaning} \\ \hline  }
\tabletail{ \hline } 
\tablelasttail{}


\chapter{Schedule Table Implementation}

\section{The States of a Schedule Table}

A schedule table always has a defined state. States include those found at page 42 of the AUTOSAR specifications 3.1 and others states used for internal management.

Indeed, \textbf{bit 1} is the "autostart" bit. It's used when autostarted schedule tables have been declared in the OIL file. Goil generates schedule tables with SCHEDULETABLE\_AUTOSTART\_X (X can be RELATIVE, ABSOLUTE or SYNCHRON) state. At startup (in tpl\_init\_os()), the system starts autostarted schedule tables and resets the \textbf{bit 1}.

\textbf{bit 4} is the "bootstrap" bit. It's used when the first expiry point of a schedule table is dated in more than \textbf{OsCounterMaxAllowedValue} ticks from the current date \footnote{As the \textit{$<$offset$>$} parameter of StartScheduleTableRel() cannot be greater than \textbf{OsCounterMaxAllowedValue} minus the \textbf{InitialOffset} of the schedule table (OS276), the first expiry point cannot be in more than \textbf{OsCounterMaxAllowedValue} ticks from the current date. Thus the "bootstrap" bit can set by StartScheduleTableAbs() only.}. It can happen when :
	\begin{itemize}
	\item the schedule table start ($<$tick\_val$>$) is after the current date and the first expiry point comes between the current date and $<$tick\_val$>$
	\item $<$tick\_val$>$ is before the current date and the first expiry point comes after the current date
	\end{itemize}

The Figure \ref{fig:bootstrapexample} below shows the bootstrap for the first item.

\begin{figure}[htbp] %  figure placement: here, top, bottom, or page
   \centering
   \includegraphics[width=6in]{pictures/BootstrapExample.pdf}  
   \caption{Bootstrap example}
   \label{fig:bootstrapexample}
\end{figure} 

The \textbf{bit 5} is the "asynchronous" bit. It tells the system that the schedule table is in asynchronous mode.\\

\begin{center}
\topcaption{\textcolor{white}{q}States of a schedule table } % Latex is weird, the table goes on the next page without the "\textcolor{white}{q}"
\begin{supertabular}{p{\Li}|c|c|c|c|c|c|p{\Lii}|} 
0	& 0	& 0	& 0 	& 0	& 0	& 0	& SCHEDULETABLE\_STOPPED  \\ 
1	& 0	& 0	& 0 	& 0	& 0	& 1	& SCHEDULETABLE\_RUNNING  \\ 
5	& 0	& 0	& 0 	& 1	& 0	& 1	& SCHEDULETABLE\_NEXT  \\  
9	& 0	& 0	& 1 	& 0	& 0	& 1	& SCHEDULETABLE\_WAITING  \\  
13	& 0	& 0	& 1 	& 1	& 0	& 1	& SCHEDULETABLE\_RUNNING\_AND\_SYNCHRONOUS \\ %\hline \hline
6	& 0	& 0	& 0 	& 1	& 1	& 0	& SCHEDULETABLE\_AUTOSTART \_ABSOLUTE  \\ 
10	& 0	& 0	& 1 	& 0	& 1	& 0	& SCHEDULETABLE\_AUTOSTART \_RELATIVE  \\  
14	& 0	& 0	& 1 	& 1	& 1	& 0	& SCHEDULETABLE\_AUTOSTART \_SYNCHRON  \\  \hline \hline
16	& 0	& 1	& 0 	& 0	& 0	& 0	& SCHEDULETABLE\_BOOTSTRAP \\ 
32	& 1	& 0	& 0 	& 0	& 0	& 0	& SCHEDULETABLE\_ASYNC  \\ 
\end{supertabular} 
\end{center}
\label{schedtablestates}

Figure  \ref{fig:STstates} shows how a schedule table goes from state to state.

\begin{figure}[htbp] %  figure placement: here, top, bottom, or page
   \centering
   \includegraphics[width=6in]{pictures/STstates.pdf}  
   \caption{States of a schedule table in Trampoline.}
   \label{fig:STstates}
\end{figure} 
	
\section{Processing a Schedule Table}

Depending on the OIL file, GOIL generates one expiry point more than the number of expiry point delared by the User : the "finalize" expiry point. Indeed, the RUNNING state of a NEXT schedule table should be set at the finalize expiry point, thus, we have to add this expiry point to do it. For a periodic schedule table, the finalize expiry point helps to launch the first expiry point of the next period. The Figure below shows the processing schedule table petri net. \\%different expiry point actions and the one we've added for the finalize expiry point.

As a schedule table is a time object, as an alarm, ...\\

%figure showing that the tpl_process_schedtable() is the main function (launched by each expiry point of a schedule table). It increment the index of the expiry point, launch the action of the  index is incremented






























%!TEX root = ./trampoline.tex

\chapter{The communication library}

\section{Internals}


%!TEX root = ./trampoline.tex

\chapter{System generation and compilation}

Trampoline is a static operating system. This means all the objects (tasks, ISR, ...) are known at compile time. This way, an application is made of tasks' code and ISRs' code, application data, and statically initialized descriptor for each object the operating system manages. A system generation tool, like \goil, generates these descriptors in C files from an application configuration described in OIL or in XML. After that the Trampoline source code, the generated files and the application source code are compiled and linked together to produce an executable file as shown in figure \ref{fig:buildtrampoline}.

\begin{figure}[htbp] %  figure placement: here, top, bottom, or page
   \centering
   \includegraphics[width=4.5in]{pictures/buildProcess.pdf} 
   \caption{\textbf{Build process of an application with Trampoline.} Starting from the left, the .c and .h corresponding to the application description given in OIL (or XML) are generated by \goil\ (or another system generation tool, for instance an Autosar compliant one) and compiled using a C compiler. Trampoline source files are compiled too and include .h from the description for configuration purpose (see section \ref{sec:configmacros}). Application files are compiled and include .h files from Trampoline. All the object files are then linked together using an optional link script generated by \goil\ or provided with the application.}\label{fig:buildtrampoline}
\end{figure}

\section{The generated files}
\label{sec:generatedfiles}

The following files are generated by \goil\ from the OIL file or should be generated if you use a different system configuration tool. More information may be found in part \ref{part:goil}.

\rowcolors{1}{white}{light-gray}
\begin{longtableii}{l|p{3.5in}}{file}{File name}{Usage}

\lineii{tpl_app_define.h\index{tpl_app_define.h}}{This file contains all the configuration macros (see section \ref{sec:configmacros}) and is included in all the Trampoline files to trigger conditional compilation. \goil\ generates this file using the \file{tpl_app_define_h.goilTemplate} template file.}

\lineii{tpl_app_config.h\index{tpl_app_config.h}}{This file contains the declarations of the constants and functions required by the OSEK and Autosar standard (like OSMAXALLOWEDVALUE_x, OSTICKSPERBASE_x or OSMINCYCLE_x constants for counter x). \goil\ generates this file using the \file{tpl_app_config_h.goilTemplate} template file.}

\lineii{tpl_app_config.c\index{tpl_app_config.c}}{This file contains the definitions of the constants and functions required by the OSEK and Autosar standard and the definitions of object descriptors used by Trampoline (see section \ref{sec:structs}) \goil\ generates this file using the \file{tpl_app_config_c.goilTemplate} template file.}

\lineii{tpl_app_custom_types.h\index{tpl_app_custom_types.h}}{Some data types used by Trampoline are not statically defined. They are generated to fit size or performance criterions. For instance, the type used for a TaskType may be a byte if there is less than 256 tasks in the system and a word otherwise. This file defined these data types.}

\lineii{tpl_service_ids.h\index{tpl_service_ids.h}}{This file is generated only if Trampoline is compiled with service calls implemented using a system call. It contains all the identifiers of the services used by the application according to the configuration. \goil\ generates this file using the \file{tpl_service_ids_h.goilTemplate} template file.}

\lineii{tpl_dispatch_table.c\index{tpl_dispatch_table.c}}{This file is generated only if Trampoline is compiled with service calls implemented using a system call. It contains the dispatch table definition. See section \ref{sec:dispatchtable}. \goil\ generates this file using the \file{tpl_dispatch_table_c.goilTemplate} template file.}

\lineii{tpl_invoque.S\index{tpl_invoque.S}}{This file is generated only if Trampoline is compiled with service calls implemented using a system call. It contains the API functions for system services. See section \ref{sec:invoque}. The extension (here .S) may change according to the assembler used. \goil\ generates this file using the \file{tpl_invoque.goilTemplate} and \file{service_call.goilTemplate} template files.}

\lineii{MemMap.h\index{MemMap.h}}{This file is generated only if memory mapping is enabled. It contains macros for compiler abstraction memory mapping of functions and data as defined in the Autosar standard \cite{autosar31memorymapping}. \goil\ generates this file using the \file{MemMap_h.goilTemplate} template file.}

\lineii{Compiler.h\index{Compiler.h}}{This file is generated only if memory mapping is enabled. It contains macros for the compiler abstraction of functions and pointer qualifier as defined in the Autosar standard \cite{autosar31compilerabstraction}. \goil\ generates this file using the \file{Compiler_h.goilTemplate} template file.}

\lineii{Compiler_Cfg.h\index{Compiler_Cfg.h}}{This file is generated only if memory mapping is enabled. It contains macros for the compiler abstraction configuration as defined in the Autosar standard \cite{autosar31compilerabstraction}. \goil\ generates this file using the \file{Compiler_Cfg_h.goilTemplate} template file.}

\lineii{script.ld\index{script.ld}}{This file is generated only if memory mapping is enabled. It contains a link script to map the executable in the target memory. \goil\ generates this file using the \file{script.goilTemplate} template file.}

\end{longtableii}

The following sections give details about the content of these files.

\section{The Configuration Macros}
\label{sec:configmacros}

Trampoline can be compiled with various options. These options are controlled by setting the appropriate preprocessor configuration macros%
\index{Configuration macros}.
These macros are usually set by \goil
using the template found in \file{tpl_app_define_h.goilTemplate} file to produce the \file{tpl_app_define.h}\index{tpl_app_define.h} file that is included by the files of Trampoline. However, a different generation tool may be used and it should comply to the specification presented in the following tables. When Trampoline is compiled, the coherency and consistency of the configuration macros are checked, by using the preprocessor macros located in the \file{tpl_config_check.h} file, to ensure they correspond to a supported configuration.

3 kinds of configuration macros are used: boolean macros, numerical macros, symbol macros and string macros. Boolean macros may take 2 values: YES or NO. All macros should be defined, Trampoline does not use the \lstinline[language=C]{#ifdef} or \lstinline[language=C]{\#ifndef} scheme to limit the occurrences of unwanted misconfigurations except to prevent multiple inclusions of the same header file.

\subsection{Number of objects macros}

These macros gives the number of objects of each kind (tasks, ISRs, resources, \ldots) and other values. They are used in Trampoline to check the validity of the various identifiers and to define tables of the corresponding size:
 
\rowcolors{1}{white}{light-gray}
\begin{longtableiii}{l|l|p{3.1in}}{var}{Macro}{Kind}{Effect}
  \lineiii{\idxconfflag{PRIO_LEVEL_COUNT}}
  {Integer}
  {The number of priority levels used in the system.}
  \lineiii{\idxconfflag{TASK_COUNT}}
  {Integer}
  {The number of tasks (basic and extended) used in the system.}
  \lineiii{\idxconfflag{EXTENDED_TASK_COUNT}}
  {Integer}
  {The number of extended tasks used in the system.}
  \lineiii{\idxconfflag{ISR_COUNT}}
  {Integer}
  {The number of ISR category 2 used in the system.}
  \lineiii{\idxconfflag{ALARM_COUNT}}
  {Integer}
  {The number of alarms used in the system.}
  \lineiii{\idxconfflag{RESOURCE_COUNT}}
  {Integer}
  {The number of resources used in the system.}
  \lineiii{\idxconfflag{SEND_MESSAGE_COUNT}}
  {Integer}
  {The number of send messages used in the system.}
  \lineiii{\idxconfflag{RECEIVE_MESSAGE_COUNT}}
  {Integer}
  {The number of receive messages used in the system.}
  \lineiii{\idxconfflag{SCHEDTABLE_COUNT}}
  {Integer}
  {The number of schedule tables used in the system. This macros is only used when WITH_AUTOSAR is set to YES.}
  \lineiii{\idxconfflag{COUNTER_COUNT}}
  {Integer}
  {The number of counters used in the system. This macros is only used when WITH_AUTOSAR is set to YES.}
  \lineiii{\idxconfflag{APP_COUNT}}
  {Integer}
  {The number of OS applications used in the system. This macros is only used when WITH_AUTOSAR is set to YES.}
  \lineiii{\idxconfflag{TRUSTED_FCT_COUNT}}
  {Integer}
  {The number of trusted functions used in the system. This macros is only used when WITH_AUTOSAR is set to YES.}
  \lineiii{\idxconfflag{RES_SCHEDULER_PRIORITY}}
  {Integer}
  {The priority of the RES_SCHEDULER resource. This should be equal to the highest priority among the tasks.}
\end{longtableiii}

\subsection{Error Handling Macros}

Error handling related macros are used to configure what kind of error Trampoline checks and what extra processing is done when an error is encountered. The following macros are available:

\rowcolors{1}{white}{light-gray}
\begin{longtableiii}{l|l|p{2.5in}}{var}{Macro}{Kind}{Effect}
  \lineiii{\idxconfflag{WITH_OS_EXTENDED}}
  {Boolean}
  {When set to YES, Trampoline system services perform error checking on their arguments. \var{WITH_OS_EXTENDED} is set to YES with a \var{STATUS} = EXTENDED and is set to NO with a \var{STATUS} = BASIC in the oil OS object.}
  \lineiii{\idxconfflag{WITH_ERROR_HOOK}\label{sec:errorhook}}
  {Boolean}
  {When set to YES, the \function{ErrorHook()} function is called if an error occurs. \var{WITH_ERROR_HOOK} is set to YES/NO with a \var{ERRORHOOK} = TRUE/FALSE in the oil OS object.}
  \lineiii{\idxconfflag{WITH_USEGETSERVICEID}}
  {Boolean}
  {When set to YES, Trampoline system services store the id of the current service. This id may be retrieved in the \function{ErrorHook()} function by using the \function{OSErrorGetServiceId()} macro. \var{WITH_USEGETSERVICEID} is set to YES/NO with a \var{USEGETSERVICEID} = TRUE/FALSE in the oil OS object.}
  \lineiii{\idxconfflag{WITH_USEPARAMETERACCESS}}
  {Boolean}
  {When set to YES, Trampoline system services store the arguments of the current service. These arguments may be retrieved in the \function{ErrorHook()} function by using the ad-hoc access macros (see \ref{sec:errorhook}). \var{WITH_USEPARAMETERACCESS} is set to YES/NO with a \var{USEPARAMETERACCESS} = TRUE/FALSE in the oil OS object.}
  \lineiii{\idxconfflag{WITH_COM_ERROR_HOOK}}
  {Boolean}
  {When set to YES, the communication error hook is called when error occurs in the communication sub-system. This macro is only available when WITH_COM is set to true.}
  \lineiii{\idxconfflag{WITH_COM_USEGETSERVICEID}}
  {Boolean}
  {When set to YES, Trampoline/COM system services store the id of the current service. This id may be retrieved in the \function{COMErrorHook()} function by using the \function{COMErrorGetServiceId()} macro. \var{WITH_COM_USEGETSERVICEID} is set to YES/NO with a \var{COMUSEGETSERVICEID} = TRUE/FALSE in the oil COM object.}
  \lineiii{\idxconfflag{WITH_COM_USEPARAMETERACCESS}}
  {Boolean}
  {When set to YES, Trampoline/COM system services store the arguments of the current service. These arguments may be retrieved in the \function{COMErrorHook()} function by using the ad-hoc access macros (see \ref{sec:comerrorhook}). \var{WITH_COM_USEPARAMETERACCESS} is set to YES/NO with a \var{COMUSEPARAMETERACCESS} = TRUE/FALSE in the oil COM object.}
  \lineiii{\idxconfflag{WITH_COM_EXTENDED}}
  {Boolean}
  {When set to YES, Trampoline/COM system services perform error checking on their arguments. \var{WITH_COM_EXTENDED} is set to YES with a \var{COMSTATUS} = EXTENDED and is set to NO with a \var{COMSTATUS} = BASIC in the oil COM object.}
\end{longtableiii}

\subsection{Protection Macros}

Protection macros deal with protection facility provided by the Autosar standard. The following Macros are available:

\rowcolors{1}{white}{light-gray}
\begin{longtableiii}{l|l|p{2.9in}}{var}{Macro}{Kind}{Effect}
  \lineiii{\idxconfflag{WITH_MEMORY_PROTECTION}}
  {Boolean}
  {When set to YES, Trampoline enables the memory protection facility. This is only supported on some ports (MPC5510 and ARM9 at time of writing). Memory protection requires the memory mapping and the use of system call. \var{WITH_MEMORY_PROTECTION} is set to YES/NO with the MEMORY_PROTECTION attribute of MEMMAP object (see \ref{sec:memmap}) set to TRUE/FALSE.}
  \lineiii{\idxconfflag{WITH_TIMING_PROTECTION}}
  {Boolean}
  {When set to YES, Trampoline enables the timing protection facility. \var{WITH_TIMING_PROTECTION} is set to YES if the \var{AUTOSAR_SC} is 2 or 4 (see \ref{sec:autosarsc}) and a least one of the objects specifies a timing protection related attribute in the oil file.}
  \lineiii{\idxconfflag{WITH_PROTECTION_HOOK}\label{sec:protectionhook}}
  {Boolean}
  {When set to YES, Trampoline calls the ProtectionHook() with the appropriate argument when a protection fault occurs. \var{WITH_PROTECTION_HOOK} is set to YES with a \var{PROTECTIONHOOK} = TRUE in the oil OS object.}
  \lineiii{\idxconfflag{WITH_STACK_MONITORING}}
  {Boolean}
  {When set to YES, Trampoline enables the stack monitoring. Each time a context switch occurs, the stack pointer is checked. If the stack pointer is outside the stack zone of the process, a fault occurs. \var{WITH_STACK_MONITORING} is set to YES with a \var{STACKMONITORING} = TRUE in the oil OS object.}
\end{longtableiii}

\subsection{Hook call macros}

Hook call macros control whether a hook is called or not. The following Macros are available:

\rowcolors{1}{white}{light-gray}
\begin{longtableiii}{l|l|p{3.1in}}{var}{Macro}{Kind}{Effect}
  \lineiii{\idxconfflag{WITH_ERROR_HOOK}}
  {Boolean}
  {see \ref{sec:errorhook}}
  \lineiii{\idxconfflag{WITH_PRE_TASK_HOOK}}
  {Boolean}
  {When set to YES, each time a task is scheduled, the function PreTaskHook() is called.
  \var{WITH_PRE_TASK_HOOK} is set to YES/NO with a \var{PRETASKHOOK} = TRUE/FALSE
  in the oil OS object.}
  \lineiii{\idxconfflag{WITH_POST_TASK_HOOK}}
  {Boolean}
  {When set to YES, each time a task is descheduled, the function PostTaskHook() is called.
  \var{WITH_POST_TASK_HOOK} is set to YES/NO with a \var{POSTTASKHOOK} = TRUE/FALSE
  in the oil OS object.}
  \lineiii{\idxconfflag{WITH_STARTUP_HOOK}}
  {Boolean}
  {When set to YES, the function StartupHook() is called within the StartOS service.
  \var{WITH_STARTUP_HOOK} is set to YES/NO with a \var{STARTUPHOOK} = TRUE/FALSE
  in the oil OS object.}
  \lineiii{\idxconfflag{WITH_SHUTDOWN_HOOK}}
  {Boolean}
  {When set to YES, the function ShutdownHook() is called within the ShutdownOS service.
  \var{WITH_SHUTDOWN_HOOK} is set to YES/NO with a \var{SHUTDOWNHOOK} = TRUE/FALSE
  in the oil OS object.}
  \lineiii{\idxconfflag{WITH_PROTECTION_HOOK}}
  {Boolean}
  {see \ref{sec:protectionhook}}
\end{longtableiii}

\subsection{Miscellaneous macros}

Here are the other available macros:

\rowcolors{1}{white}{light-gray}
\begin{longtableiii}{l|l|p{2.6in}}{var}{Macro}{Kind}{Effect}
  \lineiii{\idxconfflag{WITH_USERESSCHEDULER}}
  {Boolean}
  {When set to YES, the RES_SCHEDULER resource is used by at least one process. \var{WITH_USERESSCHEDULER} is set to YES/NO with a \var{USERESSCHEDULER} = TRUE/FALSE
  in the oil OS object.}
  \lineiii{\idxconfflag{WITH_SYSTEM_CALL}}
  {Boolean}
  {When set to YES, services are called by the mean of a system call, also known as a software interrupt (see section \ref{sec:systemcall}).
  \var{WITH_SYSTEM_CALL} is set to YES/NO according to the target architecture and requires a memory mapping}
  \lineiii{\idxconfflag{WITH_MEMMAP}}
  {Boolean}
  {When set to YES, a memory mapping is used: A \file{MemMap.h} files giving the available memory segments is included and should be generated or provided by the user. \goil\ generates such a file.
  \var{WITH_MEMMAP} is set to YES/NO with a \var{MEMMAP} = TRUE/FALSE
  in the oil OS object.}
  \lineiii{\idxconfflag{WITH_COMPILER_SETTINGS}}
  {Boolean}
  {When set to YES, the compiler dependent Autosar macros are used: \file{Compiler.h} and \file {Compiler_Cfg.h} files are includes and should generated or provided by the user. \goil\ generates these  files if \var{MEMMAP} is TRUE and the \var{COMPILER} sub-attribute is set.}
  \lineiii{\idxconfflag{WITH_AUTOSAR}}
  {Boolean}
  {When set to YES, Trampoline contains additional system services, code and declarations related to the Autosar standard. For instance, the counter descriptor includes the counter type (hardware or software).
  \var{WITH_AUTOSAR} is set to YES/NO when at least an Autosar object is present in the system configuration (oil file for instance).}
  \lineiii{\idxconfflag{TRAMPOLINE_BASE_PATH}}
  {String}
  {The path to Trampoline root directory.}
  \lineiii{\idxconfflag{AUTOSAR_SC}}
  {Integer}
  {The Autosar scalability class\index{Scalability class} ranging from 0 to 4. 0 means OSEK}
  \lineiii{\idxconfflag{WITH_OSAPPLICATION}}
  {Boolean}
  {When set to YES, OS Application\index{OS Application} are used.}
  \lineiii{\idxconfflag{WITH_TRACE}}
  {Boolean}
  {When set to YES, the tracing of the operating system is enabled. Only available if WITH_TRACE is set to YES.}
  \lineiii{\idxconfflag{TRACE_TASK}}
  {Boolean}
  {When set to YES, task (de)scheduling events are traced. Only available if WITH_TRACE is set to YES.}
  \lineiii{\idxconfflag{TRACE_ISR}}
  {Boolean}
  {When set to YES, ISR category 2 (de)scheduling events are traced. Only available if WITH_TRACE is set to YES.}
  \lineiii{\idxconfflag{TRACE_RES}}
  {Boolean}
  {When set to YES, resources get and release are traced. Only available if WITH_TRACE is set to YES.}
  \lineiii{\idxconfflag{TRACE_ALARM}}
  {Boolean}
  {When set to YES, alarm activities are traced. Only available if WITH_TRACE is set to YES.}
  \lineiii{\idxconfflag{TRACE_U_EVENT}}
  {Boolean}
  {When set to YES, user events are traced. Only available if WITH_TRACE is set to YES.}
  \lineiii{\idxconfflag{TRACE_FORMAT}}
  {Symbol}
  {Trace format. A function named tpl_trace_format_\toreplace{trace_format} is expected. Only available if WITH_TRACE is set to YES.}
  \lineiii{\idxconfflag{TRACE_FILE}}
  {String}
  {File name where the trace is stored. Usable on Posix target only. Only available if WITH_TRACE is set to YES.}
  \lineiii{\idxconfflag{WITH_IT_TABLE}}
  {Boolean}
  {When set to YES, the external interrupts are dispatched using a table of fonction pointers.}
  \lineiii{\idxconfflag{WITH_COM}}
  {Boolean}
  {When set to YES, internal communication is used.}
  \lineiii{\idxconfflag{TPL_COMTIMEBASE}}
  {Integer}
  {The COMTIMEBASE expresses in nanoseconds.}
  \lineiii{\idxconfflag{WITH_COM_STARTCOMEXTENSION}}
  {Boolean}
  {When set to YES, the communication extension function is called.}
\end{longtableiii}

\section{Application configuration}

The application configuration is generated by \goil\ using the template found in \file{tpl_app_config_h.goilTemplate} file and \file{tpl_app_config_c.goilTemplate} file to produce the \file{tpl_app_define.h}\index{tpl_app_define.h} and \file{tpl_app_define.c}\index{tpl_app_define.c} files.

\subsection{Counter related constants declaration}

The \file{tpl_app_config.h} files contains the counters related constants: those of the SystemCounter\footnote{the default counter of an OSEK operating system} and those of the counters defined by the user. The SystemCounter constants are located in the generated files because the SystemCounter default attributes may be modified by the user in the OIL or XML file. The constants of a user defined counter are declared as follow:

\begin{lstlisting}[language=C]
extern CONST(tpl_tick, OS_CONST) OSTICKSPERBASE_<counter name>;
extern CONST(tpl_tick, OS_CONST) OSMAXALLOWEDVALUE_<counter name>;
extern CONST(tpl_tick, OS_CONST) OSMINCYCLE_<counter name>;
\end{lstlisting}

Where \toreplace{counter name} is obviously the name given to the counter in the confguration. For the SystemCounter, the following constants are declared:

\begin{lstlisting}[language=C]
extern CONST(tpl_tick, OS_CONST) OSTICKSPERBASE;
extern CONST(tpl_tick, OS_CONST) OSMAXALLOWEDVALUE;
extern CONST(tpl_tick, OS_CONST) OSMINCYCLE;
\end{lstlisting}

\subsection{Events definition}

The \file{tpl_app_config.c} file should contain the event mask definitions. For each event defined in the configuration, the following lines should appear:

\begin{lstlisting}[language=C]
#define API_START_SEC_CONST_UNSPECIFIED
#include "tpl_memmap.h"

#define <event name>_mask <mask value>
CONST(EventMaskType, AUTOMATIC) <event name> = <event name>_mask;

#define API_STOP_SEC_CONST_UNSPECIFIED
#include "tpl_memmap.h"
\end{lstlisting}

Where \toreplace{event name} is the name given to the event in the configuration and \toreplace{mask value} is the value set by the user in the configuration or, when set to AUTO, the value computed by the generation tool.

\subsection{Standard resources definition}

Standard resources need the definition of an identifier used to reference the resource in a system service (\cfunction{GetResource()} and \cfunction{ReleaseResource()}) and an instance of a \ctype{tpl_resource} structure (see \ref{sec:structtplresource}). This is done with the following definitions:

\begin{lstlisting}[language=C]
#define API_START_SEC_CONST_UNSPECIFIED
#include "tpl_memmap.h"

#define <resource name>_id <resource id>
CONST(ResourceType, AUTOMATIC) <resource name> = <resource name>_id;

#define API_STOP_SEC_CONST_UNSPECIFIED
#include "tpl_memmap.h"

#define OS_START_SEC_VAR_UNSPECIFIED
#include "tpl_memmap.h"

VAR(tpl_resource, OS_VAR) <resource name>_rez_desc = {
  /* ceiling priority of the resource */  <resource priority>,
  /* owner previous priority          */  0,
  /* owner of the resource            */  INVALID_PROC_ID,
#if WITH_OSAPPLICATION == YES
  /* OS Application id                */  <resource application id>,
#endif    
  /* next resource in the list        */  NULL
};

#define OS_STOP_SEC_VAR_UNSPECIFIED
#include "tpl_memmap.h"
\end{lstlisting}

Where \toreplace{resource name} is the name given to the resource in the configuration, \toreplace{resource priority} is the priority of the resource that is computed by the generation tool and is the maximum priority of the processes that use the resource and \toreplace{resource application id} is the identifier of the OS Application the resource belongs to. Since this field is protected by \var{WITH_OSAPPLICATION}, it may be leaved empty when no OS Application is used.

\toreplace{resource id} ranges from 0 to the number of standard resources minus 1. Once every standard resource descriptor is defined, a table gathering pointers to the resource descriptors and indexed by the resource id has to be defined. This table is used by system services to get the resource descriptor from the resource id. Suppose 3 standard resource, \var{motor1}, \var{motor2} and \var{dac} has been defined and RES_SCHEDULER is used, the table should be as follow:

\begin{lstlisting}[language=C]
#define OS_START_SEC_CONST_UNSPECIFIED
#include "tpl_memmap.h"
CONSTP2VAR(tpl_resource, AUTOMATIC, OS_APPL_DATA)
tpl_resource_table[RESOURCE_COUNT] = {
  &motor1_rez_desc,
  &motor2_rez_desc,
  &dac_rez_desc,
  &res_sched_rez_desc  
};
#define OS_STOP_SEC_CONST_UNSPECIFIED
#include "tpl_memmap.h"
\end{lstlisting}

\var{\&res_sched_rez_desc}, the pointer to the resource descriptor of RES_SCHEDULER should always be the last element of the table. If RES_SCHEDULER is not used, simply remove it from the table.

\subsection{Tasks definition}

Each task needs an identifier to reference a task un a system service (\cfunction{ActivateTask()}, \cfunction{ChainTask()}, \cfunction{GetTaskState()}, \cfunction{SetEvent()} and \cfunction{GetEvent()}) and the declaration of the task function. The following definitions should appear for each task:

\begin{lstlisting}[language=C]
#define API_START_SEC_CONST_UNSPECIFIED
#include "tpl_memmap.h"

#define <task name>_id <task id>
CONST(TaskType, AUTOMATIC) <task name> = <task name>_id;

#define API_STOP_SEC_CONST_UNSPECIFIED
#include "tpl_memmap.h"

#define APP_Task_<task name>_START_SEC_CODE
#include "tpl_memmap.h"

FUNC(void, OS_APPL_CODE) <task name>_function(void);

#define APP_Task_<task name>_STOP_SEC_CODE
#include "tpl_memmap.h"
\end{lstlisting}

Where \toreplace{task name} is the name given to the task in the configuration and \toreplace{task id} is the identifier of the task computed by the system generation tool. Task ids should range from 0 to the number of tasks minus 1. In addition, id allocation must start with extended tasks first and basic task after. In addition an instance of the static task descriptor must be provided:

\begin{lstlisting}[language=C]
#define OS_START_SEC_CONST_UNSPECIFIED
#include "tpl_memmap.h"
CONST(tpl_proc_static, OS_CONST) <task name>_task_stat_desc = {
  /* context                  */  <task name>_CONTEXT,
  /* stack                    */  <task name>_STACK,
  /* entry point (function)   */  <task name>_function,
  /* internal ressource       */  <internal resource>,
  /* task id                  */  <task name>_id,
#if WITH_OSAPPLICATION == YES
  /* OS application id        */  <application>,
#endif
  /* task base priority       */  <task priority>,
  /* max activation count     */  <task activation>,
  /* task type                */  <task type>
#if WITH_AUTOSAR_TIMING_PROTECTION == YES
  /* pointer to the timing
     protection descriptor    */  ,<timing protection>
#endif
};
#define OS_STOP_SEC_CONST_UNSPECIFIED
#include "tpl_memmap.h"
\end{lstlisting}

Where \toreplace{task name} is the name given to the task in the configuration. \toreplace{internal resource} mays be one of the following:

\begin{itemize}
\item a pointer to the internal resource descriptor (see \ref{sec:internalresource}) if an internal resource has been defined in the configuration;
\item a pointer to the scheduler internal resource if the task has been defined as non-preemptable in the configuration;
\item NULL if none of the above cases apply.
\end{itemize}

\toreplace{application} is the id of the OS Application the task belongs to when OS Application are used or, when they are not used, nothing at all. \toreplace{task priority} is the priority of the task as computed by the \sysgen. \toreplace{task activation} is the maximum number of task activation allowed as defined in the configuration. \toreplace{task type} may be EXTENDED or BASIC. \toreplace{timing protection} is a pointer to the timing protection descriptor or NULL if no timing protection is defined for the task.

Also an instance of the dynamic task descriptor must be provided:

\begin{lstlisting}[language=C]
#define OS_START_SEC_VAR_UNSPECIFIED
#include "tpl_memmap.h"

VAR(tpl_proc, OS_VAR) <task name>_task_desc = {
  /* resources                      */  NULL,
#if WITH_MEMORY_PROTECTION == YES
  /* if > 0 the process is trusted  */  <trusted count>,    
#endif /* WITH_MEMORY_PROTECTION */
  /* activate count                 */  0,
  /* task priority                  */  <task priority>,
  /* task state                     */  <task state>
#if WITH_AUTOSAR_TIMING_PROTECTION == YES
  /* activation allowed             */  ,TRUE
#endif
};

#define OS_STOP_SEC_VAR_UNSPECIFIED
#include "tpl_memmap.h"
\end{lstlisting}

Where \toreplace{task name} is the name given to the task in the configuration. \toreplace{trusted count} is 0 if the task belongs to a non trusted OS Application and 1 if the tasks belongs to a trusted OS Application. \toreplace{task priority} is the priority of the task as computed by the \sysgen. \toreplace{task state} is the initial state of the task and must be set to AUTOSTART or SUSPENDED.

If the task is an EXTENDED one, an event mask descriptor is added:

%\lstdefinelanguage[trampconf]{C}{morecomment=[s][\color{red}]{<}{>}}

\begin{lstlisting}[language=C]
VAR(tpl_task_events, OS_VAR) <task name>_task_evts = {
  /* event set  */ 0,
  /* event wait */ 0
};
\end{lstlisting}

Where \toreplace{task name} is the name given to the task in the configuration.


%!TEX root = ./trampoline.tex

\chapter{Ports details}

\section{PowerPC}

\subsection{The MPC5510 Memory Protection Unit}

The access control rights of the memory region descriptor rules the access of 5 bus masters (labeled from 4 to 0). Unused bus masters are set to the same access right for all the regions. Bus master 4 is used for factory testing only, so the access rights should be set to no access. Bus master 3 is the Flexray controller. Since it is not used in the current version of Trampoline, it is set to no access to. Bus master 2 is the DMA controller. For the same reason it is set to no access. Bus master 1 is the Z0 core. Again it is set to no access.

The access control rights register has the following bit usage:

\includegraphics[width=\textwidth]{pictures/MPUacr.pdf} 

Bus master 4 is a special case. The 2 bits have the following meaning:

\rowcolors{1}{white}{light-gray}
\begin{longtableii}{p{0.4in}|p{4.8in}}{}{Bit}{Meaning}
  \lineii{M4RE}
  {If set to 1, bus master 4 may \strong{read} memory in the region. If 0, no read is allowed}
  \lineii{M4WE}
  {If set to 1, bus master 4 may \strong{write} memory in the region. If 0, no write is allowed}
\end{longtableii}

So in our case, these bits are set to 0.

Of course, other bus masters have more sophisticated access right:

\rowcolors{1}{white}{light-gray}
\begin{longtableii}{p{0.4in}|p{4.8in}}{}{Bit}{Meaning}
  \lineii{MxPE}
  {The PID Enable bit. Set to 0 in our case}
  \lineii{MxSM}
  {These 2 bits rules the supervisor mode access control with the following meaning: $00=rwx$, $01=rx$, $10=rw$, $11=$ {\it same as defined by MxUM}. In our case, it is set to $00$ for code and constants and to $11$ for data.}
  \lineii{MxUM}
  {These 2 bits rules the user mode access control. The first bit means $r$, the second one $w$ and the third one $x$.}
\end{longtableii}

Trampoline uses 4 descriptors:

\rowcolors{1}{white}{light-gray}
\begin{longtableiii}{p{0.8in}|p{2in}|p{2.3in}}{}{Descriptor}{Usage}{MxUM value}
  \lineiii{MPU\_RGD0}
  {Constants and program\footnote{This region is set to the whole addressing space. This is not definitive and should be improved because reading a peripheral control register should be protected. So an additional descriptor has to be used to allow the kernel (supervisor mode) a complete access on all the memory space and no access at all for applications (user mode).}.}
  {$rwx=00$ for supervisor mode, $rx=101$ for user mode.}
  \lineiii{MPU\_RGD1}
  {Private variables of the process.}
  {$rw=110$ for supervisor and user mode.}
  \lineiii{MPU\_RGD2}
  {Stack of the process.}
  {$rw=110$ for supervisor and user mode.}
  \lineiii{MPU\_RGD3}
  {Variables of the OS Application if OS Applications are used.}
  {$rw=110$ for supervisor and user mode.}
\end{longtableiii}

So values of access control bits should be:

\includegraphics[width=\textwidth]{pictures/MPUprog.pdf} 

So in hexa:

\rowcolors{1}{white}{light-gray}
\begin{longtableii}{l|l}{}{Kind}{Value}
  \lineii{Program region access}
  {$0x00000005$}
  \lineii{Variable region access}
  {$0x0000001E$}
\end{longtableii}

\subsubsection{What happen in case of memory protection violation ?}

Two exception handler are used to handle a memory protection violation, one for data access, one for code access.

The Data Storage exception is tied to the IVOR~2 vector (VPR offset = 0x020), see page 8-2 of the {\em MPC5510 Microcontroller Family Reference Manual}.

The Instruction Storage exception is tied to the IVOR~3 vector (VPR offset = 0x030), see page 8-2 of the {\em MPC5510 Microcontroller Family Reference Manual}.

However, it appears one of these exceptions is raised when the processor is in user mode. The behavior is different in supervisor mode {\em to be completed}.



\part{The Goil system generator}

%!TEX root = ./main.tex

\chapter{The Goil templates}
\label{chap:goiltemplates}

\lettrine{G}oil includes a template interpreter which is used for file generation. Goil generates the structures needed by trampoline to compile the application and may generate other files like a memory mapping file \file{MemMap.h}, the compiler abstraction files, \file{Compiler.h} and \file{Compiler\_cfg.h} and a linker script depending on which attributes you set in the OIL file. 

A template is a file which is located in the default template directory (set with the environment variable \envvar{GOIL\_TEMPLATES} or with the \longprogramopt{templates} option on the command line) or in the directory of your project. Goil starts by looking for a template in the directory of your project, then, if the template is not found, in the default templates directory.

Four sets of templates are used:
\begin{itemize}
\item code generation templates that are located in the \file{code} subdirectory of the template directory;
\item build system templates that are located in the \file{build} subdirectory;
\item compiler dependent stuff in the \file{compiler} subdirectory and
\item linker script templates in the \file{linker} subdirectory.
\end{itemize}

Templates are written using a simple language which allow to access the application configuration data and to mix them with text to produce files.

Files are produced by a template program located in the \file{root.goilTemplate} file which is as the root of the template directory. By default the following files are produced:
\begin{itemize}
\item \file{tpl\_app\_config.c} by using the \file{tpl\_app\_config.c.goilTemplate} file
\item \file{tpl\_app\_config.h} by using the \file{tpl\_app\_config.h.goilTemplate} file
\item \file{Makefile} (if option \programopt{g} or \longprogramopt{generate-makefile} is given) by using the \file{Makefile.goilTemplate} file
\item \file{script.ld} (if memory mapping is used and if the default name is not changed) by using the \file{script.goilTemplate} file
\item \file{MemMap.h} (if memory mapping is used) by using the \file{MemMap.h.goilTemplate} file
\item \file{Compiler.h} (if memory mapping is used) by using the \file{Compiler.h.goilTemplate} file
\item \file{Compiler\_Cfg.h} (if memory mapping is used) by using the\\\file{Compiler\_Cfg.h.goilTemplate} file
\end{itemize}

\section{The configuration data}

The configuration data are computed by Goil from the OIL source files, from the options on the command line and from the \file{target.cfg} file. They are available as a set of predefined boolean, string, integer or list variables. All these variables are in capital letters.

%\warning{Some configuration data are not listed here because they are dependent on the target. For instance, the \member{STACKSIZE} data may be an attribute of each item of a \var{TASKS} list for ppc target but are missing for the c166 target.}




\warning{Some configuration data are not listed here because they depend on the target. For instance, the \member{STACKSIZE} data may be an attribute of each item of a \var{TASKS} list for ppc target but are missing for the c166 target.}

\subsection{The \var{PROCESSES}, \var{TASKS}, \var{BASICTASKS}, \var{EXTENDEDTASKS}, \var{ISRS1} and \var{ISRS2} lists}

Theses variables are lists where informations about the processes\footnote{In Trampoline, a process is a task or an ISR category 2.} used in the application are stores:

\rowcolors{1}{white}{light-gray}
\begin{longtable}[c]{l|p{4.5in}}
{\bf List} & {\bf Content}\\
\hline
\var{PROCESSES} & the list of processes. The items are sorted in the following order: extended tasks, then basic tasks, then ISRs category 2.\\
\var{TASKS} & the list of tasks, basic and extended. The items are sorted in the following order: extended tasks, then basic tasks.\\
\var{BASICTASKS} & the list of basic tasks.\\
\var{EXTENDEDTASKS} & the list of extended tasks.\\
\var{ISRS1} & the list of ISR category 1.\\
\var{ISRS2} & the list of ISR category 2.\\
\end{longtable}

Each item of these lists has the following attributes:

\rowcolors{1}{white}{light-gray}
\begin{longtable}{>{\mem}l|l|p{3.56in}}
{\bf Item}&{\bf Type}&{\bf Content}\\
\hline
\endhead
{NAME} & string & the name of the process.\\
{PROCESSKIND} & string & the kind of process: \stringlit{Task} or \stringlit{ISR}.\\
{EXTENDEDTASK} & boolean & true if the process is an extended task, false otherwise.\\
{NONPREEMPTABLE} & boolean & true if the process is a non-preemptable task, false otherwise.\\
{PRIORITY} & integer & the priority of the process.\\
{ACTIVATION} & integer & the number of activation of a task. 1 for and extended task or an ISR.\\
{AUTOSTART} & boolean & true if the process is an autostart task, false otherwise.\\
{USEINTERNALRESOURCE} & boolean & true if the process is a task that uses an internal resource, false otherwise.\\
{INTERNALRESOURCE} & string & the name of the internal resource if the process is a task that uses an internal resource, empty string otherwise.\\
{RESOURCES} & list & The resources used by the process. Each item has the following attribute: \member{NAME}\\
{TIMINGPROTECTION} & struct & The timing protection attributes. This attribute does not exist if no timing protection is defined for the process. See below for the content of this struct.\\
\end{longtable}

The \var{TIMINGPROTECTION} struct has the following sub-attributes:

\rowcolors{1}{white}{light-gray}
\begin{longtable}{l|l|p{3.355in}}
{\bf Item} & {\bf Type} & {\bf Content}\\
\hline
\member{EXECUTIONBUDGET} & integer & The execution budget of a task. This attribute is not defined for an ISR.\\
\member{EXECUTIONTIME} & integer & The execution time of an ISR. This attribute is not defined for a Task.\\
\member{TIMEFRAME} & integer & The time frame.\\
\member{MAXOSINTERRUPTLOCKTIME} & integer & The maximum locking time of OS interrupts.\\
\member{MAXALLINTERRUPTLOCKTIME} & integer & The maximum locking time of all interrupts.\\
\member{RESOURCESLOCK} & list & The maximum locking time of resources.\\
\end{longtable}

Each element of the \var{RESOURCESLOCK} list has the following attributes:

\rowcolors{1}{white}{light-gray}
\begin{longtable}{l|l|p{4.05in}}
{\bf Item} & {\bf Type} & {\bf Content}\\
\hline
\member{RESOURCENAME} & string & The name of the locked resource.\\
\member{LOCKTIME} & integer & The maximum locking time of the resource.\\
\end{longtable}

\subsection{The \var{COUNTERS}, \var{HARDWARECOUNTERS} and \var{SOFTWARECOUNTERS} lists}

These list contains all the informations about the counters used in the application, including the \var{SystemCounter}.

\rowcolors{1}{white}{light-gray}
\begin{longtable}{>{\va}l|p{4.37in}}
{\bf List} & {\bf Content}\\
\endhead
\hline
{COUNTERS} & the list of counters, both hardware and software as long as the \var{SystemCounter}.\\
{HARDWARECOUNTERS} & the list of hardware counters including the \var{SystemCounter}.\\
{SOFTWARECOUNTERS} & the list of software counters.\\
\end{longtable}

Each item of this list has the following attributes:

\rowcolors{1}{white}{light-gray}
\begin{longtable}{l|l|p{3.85in}}
{\bf Item} & {\bf Type} & {\bf Content}\\
\hline
\endhead
\member{NAME} & string & the name of the counter.\\
\member{TYPE} & string & the type: \stringlit{HARDWARE_COUNTER} or \stringlit{SOFTWARE_COUNTER}.\\
\member{MAXALLOWEDVALUE} & integer & the maximum allowed value of the counter.\\
\member{MINCYCLE} & integer & the minimum cycle value of the counter.\\
\member{TICKPERBASE} & integer & the number of ticks needed to increment the counter.\\
\member{SOURCE} & string & the interrupt source name of the counter. This is be used to wrap interrupt vector to a counter incrementation function.\\
\end{longtable}

\subsection{The \var{EVENTS} list}

This list contains the informations about the events of the application. Each item has the following attributes:

\rowcolors{1}{white}{light-gray}
\begin{longtable}{l|l|p{4.52in}}
{\bf Item}&{\bf Type}&{\bf Content}\\
\hline
\member{NAME} & string & the name of the event.\\
\member{MASK} & integer & the mask of the event.\\
\end{longtable}

\subsection{The \var{ALARMS} list}

This list contains the informations about the alarms of the application. Each item has the following attributes:

\rowcolors{1}{white}{light-gray}
\begin{longtable}{l|l|p{4.2in}}
{\bf Item}&{\bf Type}&{\bf Content}\\
\hline
\member{NAME} & string & the name of the alarm.\\
\member{COUNTER} & string & the name of the counter that drives the alarm.\\
\member{ACTION} & string & the action to be done when the alarm expire. It can take the following values: \stringlit{setEvent}, \stringlit{activateTask} and \stringlit{callback}. The last action is not available in \autosar\ mode.\\
\member{TASK} & string & the name of the task on which the action is performed. This attribute is defined for \stringlit{setEvent} and \stringlit{activateTask} actions only.\\
\member{EVENT} & string & the name of the event to set on the target task. This attribute is defined for \stringlit{setEvent} action only.\\
\member{AUTOSTART} & boolean & true if the alarm is autostart, false otherwise \\
\member{ALARMTIME} & integer & the alarm time of the alarm. This attribute is set if \member{AUTOSTART} is true.\\
\member{CYCLETIME} & integer & the cycle time of the alarm. This attribute is set if \member{AUTOSTART} is true.\\
\member{APPMODE} & string & the application mode in which the alarm is autostart. This attribute is set if \member{AUTOSTART} is true.\\
\end{longtable}

\subsection{The \var{REGULARRESOURCES} and \var{INTERNALRESOURCES} lists}

These lists contains the informations about the resources of the application.

\rowcolors{1}{white}{light-gray}
\begin{longtable}{>{\va}l|p{4.31in}}
{\bf List}&{\bf Content}\\
\hline
\endhead
{REGULARRESOURCES} & the list of \oilattr{STANDARD} and \oilattr{LINKED} resources.\\
{INTERNALRESOURCES} & the list of \oilattr{INTERNAL} resources.\\
\end{longtable}

Each item has the following attributes:

\rowcolors{1}{white}{light-gray}
\begin{longtable}{>{\mem}l|l|p{4.25in}}
{\bf Item} & {\bf Type} & {\bf Content}\\
\hline
\endhead
{NAME} & string & the name of the resource.\\
{PRIORITY} & integer & the priority of the resource.\\
{TASKUSAGE} & list & the list of tasks that use the resource. Each item of this list has an attribute \member{NAME} which is the name of the task.\\
{ISRUSAGE} & list & the list of ISRs that use the resource. Each item of this list has an attribute \member{NAME} which is the name of the ISR.\\
\end{longtable}

\subsection{The \var{MESSAGES}, \var{SENDMESSAGES} and \var{RECEIVEMESSAGES} lists}

These lists contain the informations about the messages of the application.

\rowcolors{1}{white}{light-gray}
\begin{longtable}{>{\va}l|p{4.44in}}
{\bf List} & {\bf Content}\\
\hline\endhead
MESSAGES & the list of messages, both send and receive message.\\
SENDMESSAGES & the list of send messages.\\
RECEIVEMESSAGES & the list of receive messages.\\
\end{longtable}

Each item has the following attributes

\rowcolors{1}{white}{light-gray}
\begin{longtable}{>{\mem}l|l|p{3.85in}}
{\bf Item}& {\bf Type} & {\bf Content}\\
\hline\endhead
NAME & string & the name of the message.\\
MESSAGEPROPERTY & string & the type of the message. It can be \stringlit{RECEIVE\_ZERO\_INTERNAL}, \stringlit{RECEIVE\_UNQUEUED\_INTERNAL}, \stringlit{RECEIVE\_QUEUED\_INTERNAL}, \stringlit{SEND\_STATIC\_INTERNAL}, \stringlit{SEND\_ZERO\_INTERNAL} or \stringlit{RECEIVE\_ZERO\_SENDERS}.\\
{NEXT} & {string} & {the name of the next message in a receive message chain. This attribute is defined for receive messages only.}\\
{SOURCE} & {string} & {the name of the send message which is connected to the receive message. This attribute is defined for receive messages only.}\\
{CTYPE} & {string} & {the C language type of the message. This attribute is not defined for \stringlit{RECEIVE\_ZERO\_INTERNAL} and  \stringlit{SEND\_ZERO\_INTERNAL} messages.}\\
{INITIALVALUE} & {string} & {initial value of the receive message. This attribute is defined for \stringlit{RECEIVE\_UNQUEUED\_INTERNAL} and  \stringlit{RECEIVE\_ZERO\_SENDERS} messages only.}\\
{QUEUESIZE} & {integer} & {queue size of a receive queued message. This attribute is defined for \stringlit{RECEIVE\_QUEUED\_INTERNAL} messages only.}\\
{TARGET} & {string} & {target message of a send message. This is the first message in a receive message chain. This attribute is defined for \stringlit{SEND\_STATIC\_INTERNAL} and \stringlit{SEND\_ZERO\_INTERNAL} messages only.}\\
{FILTER} & {string} & {the kind of filter to apply. This attribute may take the following values: \stringlit{ALWAYS}, \stringlit{NEVER}, \stringlit{MASKEDNEWEQUALSX}, \stringlit{MASKEDNEWDIFFERSX}, \stringlit{NEWISEQUAL}, \stringlit{NEWISDIFFERENT}, \stringlit{MASKEDNEWEQUALSMASKEDOLD}, \stringlit{MASKEDNEWDIFFERSMASKEDOLD}, \stringlit{NEWISWITHIN}, \stringlit{NEWISOUTSIDE}, \stringlit{NEWISGREATER}, \stringlit{NEWISLESSOREQUAL}, \stringlit{NEWISLESS}, \stringlit{NEWISGREATEROREQUAL} or \stringlit{ONEEVERYN}.}\\
{MASK} & {integer} & {Mask of the filter when needed. This attribute is defined for \stringlit{MASKEDNEWEQUALSX}, \stringlit{MASKEDNEWDIFFERSX}, \stringlit{MASKEDNEWEQUALSMASKEDOLD} and \stringlit{MASKEDNEWDIFFERSMASKEDOLD} filters only.}\\
{X} & {integer} & {Value of the filter when needed. This attribute is defined for \stringlit{MASKEDNEWEQUALSMASKEDOLD} and \stringlit{MASKEDNEWDIFFERSX} filters only.}\\
{MIN} & {integer} & {Minimum value of the filter when needed. This attribute is defined for \stringlit{NEWISWITHIN} and \stringlit{NEWISOUTSIDE} filters only.}\\
{MAX} & {integer} & {Maximum value of the filter when needed. This attribute is defined for \stringlit{NEWISWITHIN} and \stringlit{NEWISOUTSIDE}.}\\
{PERIOD} & {integer} & {Period of the filter. This attribute is defined for \stringlit{ONEEVERYN} filter only.}\\
{OFFSET} & {integer} & {Offset of the filter. This attribute is defined for \stringlit{ONEEVERYN} filter only.}\\
{ACTION} & {string} & {the action (or notification) to be done when the message is delivered. It can take the following values: \stringlit{setEvent} or \stringlit{activateTask}.}\\
{TASK} & {string} & {the name of the task on which the notification is performed. This attribute is defined for \stringlit{setEvent} and \stringlit{activateTask} actions only.}\\
{EVENT} & {string} & {the name of the event to set on the target task. This attribute is defined for \stringlit{setEvent} notification only.}\\
\end{longtable}

\subsection{The \var{SCHEDULETABLES} list}

This list contains the informations about the schedule tables of the application.

\rowcolors{1}{white}{light-gray}
\begin{longtable}{>{\mem}l|l|p{3.985in}}
{\bf Item} & {\bf Type} & {\bf Content}\\
\hline\endhead
 {NAME}&
  {string}&
  {the name of the schedule table.}\\
 {COUNTER}&
  {string}&
  {the name of the counter which drives the schedule table.}\\
 {PERIODIC}&
  {boolean}&
  {true if the schedule table is a periodic one, false otherwise.}\\
 {SYNCSTRATEGY}&
  {string}&
  {the synchronization strategy of the schedule table. This attribute may take the following values: \stringlit{SCHEDTABLE_NO_SYNC}, \stringlit{SCHEDTABLE_IMPLICIT_SYNC} or \stringlit{SCHEDTABLE_EXPLICIT_SYNC}.}\\
 {PRECISION}&
  {integer}&
  {the precision of the synchronization. This attribute is define when \member{SYNCSTRATEGY} is \stringlit{SCHEDTABLE_EXPLICIT_SYNC}.}\\
 {STATE}&
  {string}&
  {the state of the schedule table. This attribute may take the following values: {\stringlit{SCHEDULETABLE_STOPPED}, \stringlit{SCHEDULETABLE_AUTOSTART_SYNCHRON}, \stringlit{SCHEDULETABLE_AUTOSTART_RELATIVE}} or {\stringlit{SCHEDULETABLE_AUTOSTART_ABSOLUTE}}.}\\
 {DATE}&
  {integer}&
  {the start date of the schedule table. This attribute has an actuel value when \member{STATE} is {\stringlit{SCHEDULETABLE_AUTOSTART_RELATIVE}} or {\stringlit{SCHEDULETABLE_AUTOSTART_ABSOLUTE}}, otherwise it is set to 0.}\\
 {LENGTH}&
  {integer}&
  {The length of the schedule table.}\\
 {EXPIRYPOINTS}&
  {list}&
  {The expiry points of the schedule table. See the following table for items attributes.}\\
\end{longtable}

Each item of the \member{EXPIRYPOINTS} list has the following attributes:

\rowcolors{1}{white}{light-gray}
\begin{longtable}{>{\mem}l|l|p{3.91in}}
{\bf Item}&{\bf Type}&{\bf Content}\\
\hline\endhead
 {ABSOLUTEOFFSET}&
  {integer}&
  {the absolute offset of the expiry points.}\\
 {RELATIVEOFFSET}&
  {integer}&
  {the relative offset of the expiry points from the previous expiry point.}\\
 {MAXRETARD}&
  {integer}&
  {maximum retard to keep the schedule table synchronous.}\\
 {MAXADVANCE}&
  {integer}&
  {maximum advance to keep the schedule table synchronous.}\\
 {ACTIONS}&
  {list}&
  {the actions to perform on the expiry point. See the following table for items attributes.}\\
\end{longtable}

Each item of the \member{ACTIONS} list has the following attributes:

\rowcolors{1}{white}{light-gray}
\begin{longtable}{>{\mem}l|l|p{4.03in}}
{\bf Item}&{\bf Type}&{\bf Content}\\
\hline\endhead
 {ACTION}&
  {string}&
  {the action to be done when the alarm expire. It can take the following values: \stringlit{setEvent}, \stringlit{activateTask}, \stringlit{incrementCounter} and \stringlit{finalizeScheduleTable}.}\\
 {TASK}&
  {string}&
  {the name of the task on which the action is performed. This attribute is defined for \stringlit{setEvent} and \stringlit{activateTask} actions only.}\\
 {EVENT}&
  {string}&
  {the name of the event to set on the target task. This attribute is defined for \stringlit{setEvent} action only.}\\
 {TARGETCOUNTER}&
  {string}&
  {the name of the counter to increment. This attribute is defined for \stringlit{incrementCounter} action only.}\\
\end{longtable}

\subsection{The \var{OSAPPLICATIONS} list}

This list contains the informations about the OS Applications of the application.

\rowcolors{1}{white}{light-gray}
\begin{longtable}{>{\mem}l|l|p{3.135in}}
{\bf Item}&{\bf Type}&{\bf Content}\\
\hline\endhead
 {NAME}&
  {string}&
  {the name of the OS Application.}\\
 {RESTART}&
  {string}&
  {the name of the restart task. This attribute is not defined is there is no restart task for the OS Application.}\\
 {PROCESSACCESSVECTOR}&
  {string}&
  {access right for the processes}\\
 {PROCESSACCESSITEMS}&
  {string}&
  {access right for the processes as bytes in a table}\\
 {PROCESSACCESSNUM}&
  {integer}&
  {number of elements in the previous table}\\
 {ALARMACCESSVECTOR}&
  {string}&
  {access right for the alarms}\\
 {ALARMACCESSITEMS}&
  {string}&
  {access right for the alarms as bytes in a table}\\
 {ALARMACCESSNUM}&
  {integer}&
  {number of elements in the previous table}\\
 {RESOURCEACCESSVECTOR}&
  {string}&
  {access right for the resources}\\
 {RESOURCEACCESSITEMS}&
  {string}&
  {access right for the resources as bytes in a table}\\
 {RESOURCEACCESSNUM}&
  {integer}&
  {number of elements in the previous table}\\
 {SCHEDULETABLEACCESSVECTOR}&
  {string}&
  {access right for the schedule tables}\\
 {SCHEDULETABLEACCESSITEMS}&
  {string}&
  {access right for the schedule tables as bytes in a table}\\
 {SCHEDULETABLEACCESSNUM}&
  {integer}&
  {number of elements in the previous table}\\
 {COUNTERACCESSVECTOR}&
  {string}&
  {access right for the software counters}\\
 {COUNTERACCESSITEMS}&
  {string}&
  {access right for the software counters as bytes in a table}\\
 {COUNTERACCESSNUM}&
  {integer}&
  {number of elements in the previous table}\\
 {PROCESSES}&
  {list}&
  {list of the processes that belong to the OS Application. Each item has an attribute \member{NAME} which is the name of the process.}\\
 {HASSTARTUPHOOK}&
  {boolean}&
  {true if the OS Application has a startup hook.}\\
 {HASSHUTDOWNHOOK}&
  {boolean}&
  {true if the OS Application has a shutdown hook.}\\
 {TASKS}&
  {list}&
  {list of the tasks that belong to the OS Application. Each item has an attribute \member{NAME} which is the name of the task.}\\
 {ISRS}&
  {list}&
  {list of the ISRs that belong to the OS Application. Each item has an attribute \member{NAME} which is the name of the ISR.}\\
 {ALARMS}&
  {list}&
  {list of the alarms that belong to the OS Application. Each item has an attribute \member{NAME} which is the name of the alarm.}\\
 {RESOURCES}&
  {list}&
  {list of the resources that belong to the OS Application. Each item has an attribute \member{NAME} which is the name of the resource.}\\
 {REGULARRESOURCES}&
  {list}&
  {list of the standard or linked resources that belong to the OS Application. Each item has an attribute \member{NAME} which is the name of the resource.}\\
 {INTERNALRESOURCES}&
  {list}&
  {list of the internal resources that belong to the OS Application. Each item has an attribute \member{NAME} which is the name of the resource.}\\
 {SCHEDULETABLES}&
  {list}&
  {list of the schedule tables that belong to the OS Application. Each item has an attribute \member{NAME} which is the name of the schedule table.}\\
 {COUNTERS}&
  {list}&
  {list of the counters that belong to the OS Application. Each item has an attribute \member{NAME} which is the name of the counter.}\\
 {MESSAGES}&
  {list}&
  {list of the messages that belong to the OS Application. Each item has an attribute \member{NAME} which is the name of the messages.}\\
\end{longtable}

\subsection{The \var{TRUSTEDFUNCTIONS} list}

This list contains the informations about the trusted functions of the application. Each item contains one attribute only.

\rowcolors{1}{white}{light-gray}
\begin{longtable}{>{\mem}l|l|p{4.58in}}
{\bf Item}&{\bf Type}&{\bf Content}\\
\hline\endhead
 {NAME}&
  {string}&
  {the name of the trusted function.}\\
\end{longtable}

\subsection{The \var{READYLIST} list}

This list contains the informations about the ready list. Items are sorted by priority from 0 to the maximum computed priority. The only attribute of each item is the size of the queue.

\rowcolors{1}{white}{light-gray}
\begin{longtable}{>{\mem}l|l|p{4.525in}}
{\bf Item}&{\bf Type}&{\bf Content}\\
\hline\endhead
 {SIZE}&
  {integer}&
  {the size of the queue for the corresponding priority.}\\
\end{longtable}

\subsection{The \var{SOURCEFILES}, \var{CFLAGS}, \var{CPPFLAGS}, \var{ASFLAGS}, \var{LDFLAGS} and\\ \var{TRAMPOLINESOURCEFILES} lists}

The \var{SOURCEFILES} list contains the source files as found in attributes \oilattr{APP_SRC} of the OS object in the OIL file. Each item in the list has one attribute.

\rowcolors{1}{white}{light-gray}
\begin{longtable}{>{\mem}l|l|p{4.6in}}
{\bf Item}&{\bf Type}&{\bf Content}\\
\hline\endhead
 {FILE}&
  {string}&
  {the source file name.}\\
\end{longtable}

The \var{CFLAGS} list contains the flags for the C compiler as found in attributes \oilattr{CFLAGS} of the OS object in the OIL file. Each item in the list has one attribute.

\rowcolors{1}{white}{light-gray}
\begin{longtable}{>{\mem}l|l|p{4.575in}}
{\bf Item}&{\bf Type}&{\bf Content}\\
\hline\endhead
 {CFLAG}&
  {string}&
  {the C compiler flag.}\\
\end{longtable}

The \var{CPPFLAGS} list contains the flags for the C++ compiler as found in attributes \oilattr{CPPFLAGS} of the OS object in the OIL file. Each item in the list has one attribute.

\rowcolors{1}{white}{light-gray}
\begin{longtable}{>{\mem}l|l|p{4.445in}}
{\bf Item}&{\bf Type}&{\bf Content}\\
\hline\endhead
 {CPPFLAG}&
  {string}&
  {the C++ compiler flag.}\\
\end{longtable}

The \var{ASFLAGS} list contains the flags for the assembler as found in attributes \member{ASFLAGS} of the OS object in the OIL file. Each item in the list has one attribute.

\rowcolors{1}{white}{light-gray}
\begin{longtable}{>{\mem}l|l|p{4.51in}}
{\bf Item}&{\bf Type}&{\bf Content}\\
\hline\endhead
 {ASFLAG}&
  {string}&
  {the assembler flag.}\\
\end{longtable}

The \var{LDFLAGS} list contains the flags for the linker as found in attributes \member{LDFLAGS} of the OS object in the OIL file. Each item in the list has one attribute.

\rowcolors{1}{white}{light-gray}
\begin{longtable}{>{\mem}l|l|p{4.51in}}
{\bf Item}&{\bf Type}&{\bf Content}\\
\hline\endhead
 {LDFLAG}&
  {string}&
  {the linker flag.}\\
\end{longtable}

The \var{TRAMPOLINESOURCEFILES} list contains the trampoline source files used by the application. Each item in the list has two attributes.

\rowcolors{1}{white}{light-gray}
\begin{longtable}{>{\mem}l|l|p{4.31in}}
{\bf Item}&{\bf Type}&{\bf Content}\\
\hline\endhead
 {DIRECTORY}&
  {string}&
  {the directory of the source file relative to the Trampoline root directory (\file{os}, \file{com} or \file{autosar}).}\\
 {FILE}&
  {string}&
  {the source file name.}\\
\end{longtable}


\subsection{The \var{INTERRUPTSOURCES} list}

This list is extracted from the \file{target.cfg} file. Each item has the following attributes:

\rowcolors{1}{white}{light-gray}
\begin{longtable}{>{\mem}l|l|p{4.5in}}
{\bf Item}&{\bf Type}&{\bf Content}\\
\hline\endfirsthead
 {NAME}&
  {string}&
  {the name of the interrupt source. This is one of the name used in the OIL file as value for the \oilattr{SOURCE} attribute.}\\
 {NUMBER}&
  {string}&
  {the id of the interrupt source.}\\
\end{longtable}



\subsection{Scalar data}

The following scalar data are defined:

\rowcolors{1}{white}{light-gray}
\begin{sortedtable}{>{\va}l|l|p{3.535in}}
{\bf Data}&{\bf Type}&{\bf Content}
\sortline{APPNAME}
{string}
{name of executable as given in the \oilattr{APP_NAME} attribute in the OS object} 
\sortline{ARCH}
{string}
{name of the architecture. This is the first item in the target.} 
\sortline{ASSEMBLEREXE}
{string}
{name of the assembler executable used. This is the \oilattr{ASSEMBLER} attribute in the OS object. It is set to {\em as} by default. It is used for build dependent templates.} 
\sortline{ASSEMBLER}
{string}
{name of the assembler used. This is the \oilattr{ASSEMBLER} attribute in the \oilattr{MEMMAP} attribute of the OS object. It is used for assembler dependent templates.} 
\sortline{AUTOSAR}
{boolean}
{true if Trampoline is compiled with the Autosar extension.} 
\sortline{BOARD}
{string}
{name of the board. This is the third item (if any) in the target.} 
\sortline{CHIP}
{string}
{name of the chip. This is the second item (if any) in the target.} 
\sortline{COMPILEREXE}
{string}
{name of the compiler executable used. This is the \oilattr{COMPILER} attribute in the OS object. It is set to {\em gcc} by default. It is used for build dependent templates. Do not confuse with the \oilattr{COMPILER} data.} 
\sortline{COMPILER}
{string}
{name of the compiler used. This is the \oilattr{COMPILER} attribute in the \oilattr{MEMMAP} attribute of the OS object. It is used for compiler dependent templates.} 
\sortline{CPUNAME}
{string}
{name given to the OIL CPU object} 
\sortline{EXTENDED}
{boolean}
{true if Trampoline is compiled in extended error handling mode.} 
\sortline{FILENAME}
{string}
{the name of the file which will be written as the result of the computation of the current template.} 
\sortline{FILEPATH}
{string}
{the full absolute path of the file which will be written as the result of the computation of the current template.} 
\sortline{NATIVEFILEPATH}
{string}
{the full absolute path of the file which will be written as the result of the computation of the current template in native OS format.} 
\sortline{ITSOURCESLENGTH}
{integer}
{number of interrupt sources as defined in the \file{target.cfg} file.} 
\sortline{LINKEREXE}
{string}
{name of the linker executable used. This is the \oilattr{LINKER} attribute in the OS object. It is set to {\em gcc} by default. It is used for build dependent templates. Do not confuse with the \oilattr{LINKER} data.} 
\sortline{LINKER}
{string}
{name of the linker used. This is the \oilattr{LINKER} attribute in the \oilattr{MEMMAP} attribute of the OS object. It is used for linker dependent templates.} 
\sortline{LINKSCRIPT}
{string}
{name of the link script file as given in the \oilattr{MEMMAP} attribute of the OS object.} 
\sortline{MAXTASKPRIORITY}
{integer}
{the highest computed priority among the tasks.} 
\sortline{OILFILENAME}
{string}
{name of the root OIL source file} 
\sortline{PROJECT}
{string}
{name of the project. The name of the project is the \programopt{p} (or \longprogramopt{project}) value if it is set or the name of the oil file without the extension.} 
\sortline{SCALABILITYCLASS}
{integer}
{the Autosar scalability class used by the application. If Autosar is not enabled, \oilattr{SCALABILITYCLASS} is set to 0.} 
\sortline{TARGET}
{string}
{name of the target. This is the \programopt{t} (or \longprogramopt{target}) option value of goil.} 
\sortline{TEMPLATEPATH}
{string}
{path to the template root directory. This is the \longprogramopt{templates} option value of goil or the value of the \envvar{GOIL\_TEMPLATES} environment variable.} 
\sortline{TIMESTAMP}
{string}
{current date} 
\sortline{TRAMPOLINEPATH}
{string}
{path to the trampoline root directory. This is the \oilattr{TRAMPOLINE\_BASE\_PATH} attribute of the OS object. It defaults to ``..".} 
\sortline{USECOMPILERSETTINGS}
{boolean}
{true if memory mapping is enabled (Goil generates the \file{Compiler.h} and \file{Compiler_Cfg.h} files and Trampoline includes them).} 
\sortline{USEBUILDFILE}
{boolean}
{true if a build file is used for the project ie option \programopt{g} or \longprogramopt{generate-makefile} is given.} 
\sortline{USECOM}
{boolean}
{true if the application uses OSEK COM.} 
\sortline{USEERRORHOOK}
{boolean}
{true if Trampoline uses the Error Hook.} 
\sortline{USEGETSERVICEID}
{boolean}
{true if Trampoline uses the service ids access macros.} 
\sortline{USEINTERRUPTTABLE}
{boolean}
{true if the wrapping of interrupt vector to glue functions used to increment a counter or to activate an ISR2 (for instance) should be generated. The actual code generation is up to the port.} 
\sortline{USELOGFILE}
{boolean}
{true if goil generates a log file, ie option \programopt{l} or \longprogramopt{logfile} is given.}
\sortline{USEMEMORYMAPPING}
{boolean}
{true if memory mapping is enabled (Goil generates the \file{MemMap.h} file and Trampoline includes it).} 
\sortline{USEMEMORYPROTECTION}
{boolean}
{true if Trampoline uses the Memory Protection.} 
\sortline{USEOSAPPLICATION}
{boolean}
{true if Trampoline uses OS Applications.} 
\sortline{USEPARAMETERACCESS}
{boolean}
{true if Trampoline uses the parmaters access macros.} 
\sortline{USEPOSTTASKHOOK}
{boolean}
{true if Trampoline uses the Post-Task Hook.} 
\sortline{USEPRETASKHOOK}
{boolean}
{true if Trampoline uses the Pre-Task Hook.} 
\sortline{USEPROTECTIONHOOK}
{boolean}
{true if Trampoline uses the Protection Hook.} 
\sortline{USERESSCHEDULER}
{boolean}
{true if Trampoline uses the RES_SCHEDULER resource.} 
\sortline{USESHUTDOWNHOOK}
{boolean}
{true if Trampoline uses the Shutdown Hook.} 
\sortline{USESTACKMONITORING}
{boolean}
{true if Trampoline uses the Stack Monitoring.} 
\sortline{USESTARTUPHOOK}
{boolean}
{true if Trampoline uses the Startup Hook.} 
\sortline{USESYSTEMCALL}
{boolean}
{true if services are called using a System Call (i.e. a software interrupt).} 
\sortline{USETIMINGPROTECTION}
{boolean}
{true if Trampoline uses Timing Protection.} 
\sortline{USETRACE}
{boolean}
{true if tracing is enabled.} 
\end{sortedtable}


\section{The Goil template language (or GTL)}
\label{sec:goiltemplateslanguage}

A template is a text file with file extension \file{.goilTemplate}. This kind of file mixes literal text with an embedded program. Some instructions (see section \ref{outputInstruction}) in the embedded program outputs text as a result of the program execution and this text is put in place of the instructions. The resulting file is then stored.

The template interpreter starts in literal text mode. Switching from literal text mode to program mode and back to text mode is done when a \character{\%} is encountered. A literal \character{\%} and a literal \character{\textbackslash} may be used by escaping them with a \character{\textbackslash}.

\section{GTL types}

GTL supports 5 types: \strong{string}, \strong{integer}, \strong{boolean}, \strong{list} and \strong{struct}. The 4 first types have readers %\footnote{All the readers available in the corresponding Galgas types are available}
to get informations about a variable. A reader is invokes with the following syntax:

\begin{lstlisting}[language=goilTemplate]
[expression reader]
\end{lstlisting}

A struct is an aggregate of data. The `::' allows to get a member of the struct. For instance one of the member of \var{TIMINGPROTECTION} is \member{TIMEFRAME} so to get \member{TIMEFRAME}, the following syntax is used:

\begin{lstlisting}[language=goilTemplate]
TIMINGPROTECTION::TIMEFRAME
\end{lstlisting}

\subsection{string readers}

The following readers are available for string variables:

\rowcolors{1}{white}{light-gray}
\begin{longtable}{>{\ttfamily}l|l|p{2.35in}}
{\bf Item}&{\bf Type}&{\bf Meaning}\\
\hline\endhead
 {HTMLRepresentation}&
  {string}&
  {this reader returns a representation of the string suitable for an HTML encoded representation. \character{\&} is encoded by \cdata{\&amp;} , \character{"} by \cdata{\&quot;} , \character{<} by \cdata{\&lt;} and \character{>} by \cdata{\&gt;} .}\\
 {identifierRepresentation}&
  {string}&
  {this reader returns an unique representation of the string conforming to a C identifier. Any Unicode character that is not a latin letter is transformed into its hexadecimal code point value, enclosed by \character{_} characters. This representation is unique: two different strings are transformed into different C identifiers. For example: \cdata{value3} is transformed to \cdata{value_33_}; \cdata{+=} is transformed to \cdata{_2B__3D_};
\cdata{An_Identifier} is transformed to \cdata{An_5F_Identifier}.}\\
 {lowercaseString}&
  {string}&
  {this reader returns lowercased representation of the string.}\\
 {length}&
  {integer}&
  {this reader returns the number of characters in the string}\\
 {stringByCapitalizingFirstCharacter}&
  {string}&
  {if the string is empty, this reader returns the empty string; otherwise, it returns the string, the first character being replaced with the corresponding upper case character.}\\
 {uppercaseString}&
  {string}&
  {this reader returns uppercased representation of the receiver}\\
\end{longtable}

\subsection{boolean readers}

The following readers are available for boolean variables:

\rowcolors{1}{white}{light-gray}
\begin{longtable}{>{\ttfamily}l|l|p{4.025in}}
{\bf Item}&{\bf Type}&{\bf Meaning}\\
\hline\endhead
 {trueOrFalse}&
  {string}&
  {this reader returns \stringlit{true} or \stringlit{false} according to the boolean value}\\
 {yesOrNo}&
  {string}&
  {this reader returns \stringlit{yes} or \stringlit{no} according to the boolean value}\\
 {unsigned}&
  {integer}&
  {this reader returns 0 or 1 according to the boolean value}\\
\end{longtable}

\subsection{integer readers}

The following readers are available for integer variables:

\rowcolors{1}{white}{light-gray}
\begin{longtable}{>{\ttfamily}l|l|p{4.225in}}
{\bf Item}&{\bf Type}&{\bf Meaning}\\
\hline\endhead
 {string}&
  {string}&
  {This reader returns the integer value as a character string.}\\
 {hexString}&
  {string}&
  {this reader returns an hexadecimal string representation of the integer value.}\\
 {bitAtIndex}&
  {boolean}&
  {this reader takes one {\em int} argument. It returns true if the bit at the index passed as argument is set and false if it is not set. For instance {\tt let a := 3 let b := [a bitAtIndex: 0]} set {\tt b} to true because bit 0 of {\tt a} is 1}\\
 {setBitAtIndex}&
  {integer}&
  {this reader takes two arguments. The first one, the value, is a {\em boolean} and the second one, the index, an {\em int}. it returns the integer value with the bit at the index passed as second argument set at the value passed as the first argument. For instance {\tt let b := [1 setBitAtIndex: true, 4]} set {\tt b} to 17}\\
\end{longtable}

\subsection{list readers}

The following reader is available for list variables:

\rowcolors{1}{white}{light-gray}
\begin{longtable}{>{\ttfamily}l|l|p{4.395in}}
{\bf Item}&{\bf Type}&{\bf Meaning}\\
\hline\endhead
 {length}&
  {integer}&
  {this reader returns the number of objects currently in the list.}\\
\end{longtable}

\section{GTL operators}

\subsection{Unary operators}

\rowcolors{1}{white}{light-gray}
\begin{longtable}{c|l|l|p{2.56in}}
{\bf Operator}&{\bf Operand Type}&{\bf Result Type}&{\bf Meaning}\\
\hline\endhead
 {+}&
  {integer}&
  {integer}&
  {no operation.}\\
 {$\sim$}&
  {integer}&
  {integer}&
  {bitwise not.}\\
 {not}&
  {boolean}&
  {boolean}&
  {boolean not.}\\
 {exists}&
  {{\em any variable}}&
  {boolean}&
  {true if the variable is defined, false otherwise. But see below}\\
\end{longtable}

\note{A second form of \cdata{exists} is:}
 
\begin{lstlisting}[language=goilTemplate]
exists var default (expression)
\end{lstlisting}

\var{var} and {\em expression} should have the same type. If \var{var} exists, the returned value is the content of \var{var}. If it does not exist, {\em expression} is returned.


\subsection{Binary operators}

\rowcolors{1}{white}{light-gray}
\begin{longtable}{>{\ttfamily}c|l|l|p{2.47in}}
{\bf Operator}&{\bf Operands Type}&{\bf Result Type}&{\bf Meaning}\\
\hline\endhead
 {+}&
  {integer}&
  {integer}&
  {add.}\\
 {-}&
  {integer}&
  {integer}&
  {substract.}\\
 {*}&
  {integer}&
  {integer}&
  {multiply.}\\
 {/}&
  {integer}&
  {integer}&
  {divide.}\\
 {\&}&
  {integer}&
  {integer}&
  {bitwise and.}\\
 {\&}&
  {boolean}&
  {boolean}&
  {boolean and.}\\
 {$\mid$}&
  {integer}&
  {integer}&
  {bitwise or.}\\
 {$\mid$}&
  {boolean}&
  {boolean}&
  {boolean or.}\\
 {$\wedge$}&
  {integer}&
  {integer}&
  {bitwise xor.}\\
 {$\wedge$}&
  {boolean}&
  {boolean}&
  {boolean xor.}\\
 {.}&
  {string}&
  {string}&
  {string concatenation.}\\
 {$<<$}&
  {integer}&
  {integer}&
  {shift left.}\\
 {$>>$}&
  {integer}&
  {integer}&
  {shift right.}\\
 {!=}&
  {{\em any}}&
  {boolean}&
  {comparison (different).}\\
 {==}&
  {{\em any}}&
  {boolean}&
  {comparison (equal).}\\
 {$<$}&
  {integer {\em or} boolean}&
  {boolean}&
  {comparison (lower than).}\\
 {$<=$}&
  {integer {\em or} boolean}&
  {boolean}&
  {comparison (lower or equal).}\\
 {$>$}&
  {integer {\em or} boolean}&
  {boolean}&
  {comparison (greater).}\\
 {$>=$}&
  {integer {\em or} boolean}&
  {boolean}&
  {comparison (greater or equal).}\\
\end{longtable}

\subsection{Constants}

\rowcolors{1}{white}{light-gray}
\begin{longtable}{>{\ttfamily}l|l|p{4.11in}}
{\bf Constant}&{\bf Type}&{\bf Meaning}\\
\hline\endhead
 {emptyList}&
  {list}&
  {this constant is an empty list}\\
 {true}&
  {boolean}&
  {true boolean}\\
 {false}&
  {boolean}&
  {false boolean}\\
 {yes}&
  {boolean}&
  {true boolean}\\
 {no}&
  {boolean}&
  {false boolean}\\
\end{longtable}

\section{GTL instructions}

\subsection{The {\em let} instruction}

Data assignment instruction. The general form is:

\begin{lstlisting}
let var := expression
\end{lstlisting}

A second form allows to add a string to a list (only, this should be extended in the future). The string is added with the \var{NAME} attribute.

\begin{lstlisting}
let var += expression
\end{lstlisting}

\var{var} is a list and {\em expression} is a string.

The scope of a variable depends on the location where the variable is assigned the first time. For instance, in the following code:

\begin{lstlisting}
let a := 1
foreach task in TASKS do
  let b := INDEX
  let a := INDEX
end foreach
!a !b
\end{lstlisting}

Because a is assigned outside the {\tt foreach} loop, it contains the value of the last INDEX after the {\tt foreach}. Because b is assigned inside the {\tt foreach} loop, it does not exist after the loop anymore and {\tt!b} will trigger and error.


\subsection{The {\em if} instruction}

Conditional execution. The forms are:

\begin{lstlisting}
if expression then ... end if
if expression then ... else ... end if
if expression then ... elsif expression then ... end if
if expression then ... elsif expression then ... else ... end if
\end{lstlisting}    

The {\em expression} must be boolean. In the following example, the blue text (within the \%) is produced only if the \var{USECOM} boolean variable is true:

\begin{lstlisting}
if USECOM then %
#include "tpl_com.h" %
end if
\end{lstlisting}

\subsection{The {\em foreach} instruction}

This instruction iterates on the elements of a list. Each element may have many attributes that are available as variables within the {\bf do} section of the foreach loop. The simplest form is the following one

\begin{lstlisting}
foreach var in expression do ... end foreach
\end{lstlisting}

In the following example, for each element in the \var{ALARMS} list, the text between the {\tt do} and the {\tt end foreach} is produced with the \var{NAME} attribute of the current element of the \var{ALARMS} list inserted at the specified location. \var{INDEX} is not an attribute of the current element. It is generated for each element and ranges from 0 to the number of elements in the list minus 1.
\begin{lstlisting}
foreach ALARMS do
%
/* Alarm % !NAME % identifier */
#define % !NAME %_id % !INDEX %
CONST(AlarmType, AUTOMATIC) % !NAME % = % !NAME %_id;
%
end foreach
\end{lstlisting}

A more general form of the {\tt foreach} instruction is:

\begin{lstlisting}
foreach expression prefixedby string
  before ...
  do ...
  between ...
  after ...
end foreach
\end{lstlisting}

{\tt prefixedby} is optional and allows to prefix the attribute names by {\em string}. If the list is not empty, the {\tt before} section are executed once before the first execution of the {\tt do} section. The {\tt between} section is executed between the execution of the {\tt do} section.  If the list is not empty, the {\tt after} section is executed once after the last execution of the {\tt do} section.

In the following example, a table of pointers to alarm descriptors is generated:

\begin{lstlisting}
foreach ALARMS
  before %
tpl_time_obj *tpl_alarm_table[ALARM_COUNT] = {
%
  do %  &% !NAME %_alarm_desc%
  between %,
%
  after %
};
%
end foreach
\end{lstlisting}

\subsection{The {\em for} instruction}

The {\bf for} instruction iterates along a literal list of elements.

\begin{lstlisting}
for var in expression, ... , expression do
  ...
end for
\end{lstlisting}

At each iteration, {\em var} gets the value of the current {\em expression}. As in the {\tt foreach} instruction, \var{INDEX} is generated and ranges from 0 to the number of elements in the list minus 1.

\subsection{The {\em loop} instruction}

The {\bf loop} instruction is the classical integer loop. Its simplest form is:

\begin{lstlisting}
loop var from expression to expression do
  ...
end loop
\end{lstlisting}

Like in the foreach instruction, {\tt before},  {\tt between} and  {\tt after} sections may be used:

\begin{lstlisting}
loop var from expression to expression
  before ...
  do ...
  between ...
  after ...
end loop
\end{lstlisting}


\subsection{The {\em !} instruction}
\label{outputInstruction}

{\tt !} emits an expression. The form is:

\begin{lstlisting}
! expression
\end{lstlisting}

\subsection{The {\em ?} instruction}

{\tt ?} stores in a variable a number of spaces equal to the current column in the output. The form is:

\begin{lstlisting}
? var
\end{lstlisting}

\subsection{The {\em template} instruction}

The {\tt template} instruction includes the output of another template in the output of the current template. Its simplest form is the following one:

\begin{lstlisting}
template template_file_name
\end{lstlisting}

If the file {\em template\_file\_name}.goilTemplate does not exist, an error occurs. To include the output of a template without generating an error, use the following form:

\begin{lstlisting}
template if exists template_file_name
\end{lstlisting}

A third form allows to execute instructions when the included template file is not found:

\begin{lstlisting}
template if exists template_file_name or ... end template
\end{lstlisting}

At last, it is possible to search templates in a hierarchy (code, linker, compiler, build) different from the current one. For instance to include a template located in the linker hierarchy, use one of the following forms:

\begin{lstlisting}
template template_file_name in hierarchy
template if exists template_file_name in hierarchy
template if exists template_file_name in hierarchy or ... end template
\end{lstlisting}


In all cases, the included template inherits from the current variables table but works on its own local copy.

\subsection{The {\em write} instruction}

The write instruction defines a block where the template processing output is captured to be written to a file. The general form is:

\begin{lstlisting}
write to expression :
  ...
end write
\end{lstlisting}

Where {\em expression} is a string expression.

In the following example, the result of the \file{script} template is written to the link script file.

\begin{lstlisting}
if exists LINKER then
  write to PROJECT."/".LINKSCRIPT:
    template script in linker
  end write
end if
\end{lstlisting}


\subsection{The {\em error} and {\em warning} instructions}

It can be useful to generate an error or a warning if a data is not defined or if it looks strange. For instance if a target needs a \member{STACKSIZE} for a task or if the \member{STACKSIZE} is too large for a 16bit target. \strong{error} and \strong{warning} have 2 forms:

\begin{lstlisting}
error var expression
warning var expression
\end{lstlisting}

and

\begin{lstlisting}
error here expression
warning here expression
\end{lstlisting}

{\em expression} must be of type string. In the first form, \var{var} is a configuration data. The file location of this configuration may be a location in the OIL file or in the template file if the variable was assigned in the template. In the second form, \strong{here} means the current location in the template file.

In the following example an error is generated for each task with not \member{STACKSIZE} attribute in the OIL file:

\begin{lstlisting}
foreach TASKS do
  if not exists STACKSIZE then
    error NAME "STACKSIZE of Task " . NAME . " is not defined"
  end if
end foreach
\end{lstlisting}

In this second example, a warning is generated if a template is not found:

\begin{lstlisting}
template if exists interrupt_wrapping or
  warning here "interrupt_wrapping.goilTemplate not found"
end template
\end{lstlisting}

\section{Examples}

Here are examples of code generation using GTL.

\subsection{Computing the list of process ids}

\begin{lstlisting}
foreach PROCESSES do
  if PROCESSKIND == "Task" then
%
/* Task % !NAME % identifier */
#define % !NAME %_id % !INDEX %
CONST(TaskType, AUTOMATIC) % !NAME % = % !NAME %_id;
%
  else
%
/* ISR % !NAME % identifier */
#define % !NAME %_id % !INDEX 
    if AUTOSAR then
    #
    # ISR ids constants are only available for AUTOSAR
    #
%
CONST(ISRType, AUTOMATIC) % !NAME % = % !NAME %_id;
%
    end if
  end if
end foreach
\end{lstlisting}

\subsection{Computing an interrupt table}

\begin{lstlisting}
if USEINTERRUPTTABLE then
  loop ENTRY from 0 to ITSOURCESLENGTH - 1
    before
%
#define OS_START_SEC_CONST_UNSPECIFIED
#include "tpl_memmap.h"
CONST(tpl_it_vector_entry, OS_CONST)
tpl_it_table[% !ITSOURCESLENGTH %] = {
%
    do
      let entryFound := false
      foreach INTERRUPTSOURCES prefixedby interrupt_ do
        if ENTRY == interrupt_NUMBER then
          # check first for counters
          foreach HARDWARECOUNTERS prefixedby counter_ do
            if counter_SOURCE == interrupt_NAME & not entryFound then
              %  { tpl_tick_% !interrupt_NAME %, (void *)NULL }%
              let entryFound := true
            end if
          end foreach
          if not entryFound then
            foreach ISRS2 prefixedby isr2_ do
              if isr2_SOURCE == interrupt_NAME & not entryFound then
                %  { tpl_central_interrupt_handler_2, (void*)%
                !([TASKS length] + INDEX) % }%
                let entryFound := true
              end if
            end foreach
          end if
        end if
      end foreach
      if not entryFound then
        %  { tpl_null_it, (void *)NULL }%
      end if
   between %,
%
    after
%
};
#define OS_STOP_SEC_CONST_UNSPECIFIED
#include "tpl_memmap.h"
%
 end loop
end if
\end{lstlisting}

\subsection{Generation of all the files}

This is the default \file{root.goilTemplate} file

\begin{lstlisting}
write to PROJECT."/tpl_app_config.c":
  template tpl_app_config_c in code
end write

write to PROJECT."/tpl_app_config.h":
  template tpl_app_config_h in code
end write

write to PROJECT."/tpl_app_define.h":
  template tpl_app_define_h in code
end write

if exists COMPILER then
  write to PROJECT."/MemMap.h":
    template MemMap_h in compiler
  end write
  write to PROJECT."/Compiler.h":
    template Compiler_h in compiler
  end write
  write to PROJECT."/Compiler_Cfg.h":
    template Compiler_Cfg_h in compiler
  end write
end if

if exists LINKER then
  write to PROJECT."/".LINKSCRIPT:
    template script in linker
  end write
end if
\end{lstlisting}


\printindex
\bibliographystyle{plain}
\bibliography{trampoline}

 \end{document} 